\documentclass{article}
\usepackage{geometry}
\usepackage{paralist}
\usepackage[T1]{fontenc}
\usepackage{reledmac}
\usepackage{changepage}
\usepackage{layout}

\usepackage{pgfplots}
\usepackage{tikz}
\usetikzlibrary{positioning}
\usetikzlibrary{shapes.geometric, arrows}
\tikzstyle{arrow} = [thick,->,>=stealth]

\usepackage{fancyhdr}
\fancyhead[L]{
	\begin{tabular}{l}
		\Large \textbf{\textsc{Advanced Networking and Future Internet}} \\
		\large Theoretical Exercise 09
	\end{tabular}
}
\fancyhead[R]{
	\begin{tabular}{r}
		16-124-836 \\
		Marcel \textsc{Zauder}
	\end{tabular}
}
\renewcommand{\headrulewidth}{0.4pt}
\fancyfoot[C]{\thepage}
\renewcommand{\footrulewidth}{0.4pt}
\setlength{\headsep}{35pt}
\setlength{\textheight}{600pt}

\usepackage{hyperref}

\begin{document}
	\pagestyle{fancy}
	
	\section*{9.1 Uncompressed Audio \& Video}
	\begin{adjustwidth}{2em}{2em}
		\begin{enumerate}
			\item \textbf{Audio} \\
			\begin{tabular}{|c|c|c|c|}
				\hline
				\textbf{Quality} & \textbf{Sample Rate (kHz)} & \textbf{Quantization Level} & \textbf{Bandwidth (Mbps/Gbps)} \\
				\hline
				AM Radio (stereo) & 11.025 & 8 & 0.1764 Mbps \\
				FM Radio (stereo) & 22.05 & 16 & 0.7056 Mbps \\
				HD/DVD Audio & 192 & 24 & 4.608 Mbps \\
				\hline
			\end{tabular}
			\item \textbf{Video} \\
			\begin{tabular}{|c|c|c|c|c|}
				\hline
				\textbf{Quality} & \textbf{Resolution} & \textbf{Bits Per Pixel} & \textbf{FPS} & \textbf{Bandwidth (Mbps/Gbps)} \\
				\hline
				4k Video & 3840 x 2160 & 24 & 24 & 4'777.574 Mbps / 4.777 Gbps \\
				5k Video & 5140 x 2880 & 36 & 30 & 15'987.456 Mbps / 15.987 Gbps \\
				8k Video & 7680 x 4320 & 48 & 60 & 95'551.488 Mbps / 95.551 Gbps \\
				\hline
			\end{tabular}
		\end{enumerate}
	\end{adjustwidth}
	
	\section*{9.2 Digitization of Audio Data}
	\begin{adjustwidth}{2em}{2em}
	\end{adjustwidth}
	
	\section*{9.3 Source Encoding}
	\begin{adjustwidth}{2em}{2em}
	\end{adjustwidth}
	
	\section*{9.4 MPEG Audio}
	\begin{adjustwidth}{2em}{2em}
	\end{adjustwidth}
	
	\section*{9.5 Huffman Coding}
	\begin{adjustwidth}{2em}{2em}
		\begin{enumerate}
			\item Encoding of \textbf{IACEANFI}: \\
			0000 11 10 001 11 0001 01 0000
			\item Encoding of \textbf{MARCEL\_ZAUDER}: \\
			\begin{tabular}{ccc}
				\begin{tabular}{|r|r|}
					\hline
					\textbf{Symbol} & \textbf{Weight} \\
					\hline
					M & 1/13 \\
					A & 2/13 \\
					R & 2/13 \\
					C & 1/13 \\
					E & 2/13 \\
					L & 1/13 \\
					Z & 1/13 \\
					U & 1/13 \\
					D & 1/13 \\
					'space' & 1/13 \\
					\hline					
				\end{tabular}
				&
				\begin{tabular}{|r|r|}
					\hline
					\textbf{Symbol} & \textbf{Encoding} \\
					\hline
					A & 11 \\
					M & 101 \\
					E & 001 \\
					R & 000 \\
					L & 1001 \\
					C & 1000 \\
					U & 0111 \\
					Z & 0110 \\
					'space' & 0101 \\
					D & 0100 \\					
					\hline
				\end{tabular}
				&
				\begin{tabular}{p{6cm}}
					\textbf{Encoding}: \\
					101 11 000 1000 001 1001 \\ 0101 \\ 0110 11 0111 0100 001 000
				\end{tabular}
			
			\end{tabular}
			\item What do we call codes of this property?
			Code of this type is called a prefix code. A prefix code requires that ther is no whole code word in the system that is a prefix of any other code word in the system. Therefore no seperator is required.
		\end{enumerate}
	\end{adjustwidth}
	
	\section*{9.6 Pulse Code Modulation}
	\begin{adjustwidth}{2em}{2em}
	\end{adjustwidth}
\end{document}