\documentclass{article}
\usepackage{geometry}
\usepackage{paralist}
\usepackage[T1]{fontenc}
\usepackage{reledmac}
\usepackage{changepage}

\usepackage{pgfplots}
\usepackage{tikz}
\usetikzlibrary{positioning}
\usetikzlibrary{shapes.geometric, arrows}
\tikzstyle{arrow} = [thick,->,>=stealth]

\usepackage{graphicx}
\graphicspath{ {./images/} }

\usepackage{fancyhdr}
\fancyhead[L]{
	\begin{tabular}{l}
		\Large \textbf{\textsc{Advanced Networking and Future Internet}} \\
		\large Practical Exercise 01
	\end{tabular}
}
\fancyhead[R]{
	\begin{tabular}{r}
		16-124-836 \\
		Marcel \textsc{Zauder}
	\end{tabular}
}
\renewcommand{\headrulewidth}{0.4pt}
\fancyfoot[C]{\thepage}
\renewcommand{\footrulewidth}{0.4pt}

\usepackage{hyperref}

\begin{document}
	\pagestyle{fancy}
	\hfill
	
	\section*{1.4.1 How many nodes are reachable before and after starting the controller?}
	\begin{adjustwidth}{2em}{2em}
		\subsection*{1.4.1.a Before starting the controller}
		\begin{adjustwidth}{2em}{2em}
		\begin{enumerate}
			\item \textsc{linear} \\
			\includegraphics[scale=0.4]{linear_before.png}
			\item \textsc{single} \\
			\includegraphics[scale=0.4]{single_before.png}
			\item \textsc{tree} \\
			\includegraphics[scale=0.5]{tree_before.png}
		\end{enumerate}
		\end{adjustwidth}
		\subsection*{1.4.1.b After starting the controller}
		\begin{adjustwidth}{2em}{2em} 
		\begin{enumerate}
			\item \textsc{linear} \\
			\includegraphics[scale=0.5]{linear_after.png}
			\item \textsc{single} \\
			\includegraphics[scale=0.5]{single_after.png}
			\item \textsc{tree} \\
			\includegraphics[scale=0.5]{tree_after.png} \\
		\end{enumerate}
		\end{adjustwidth}
		Before setting up the switches none of the hosts can communicate with each other because the packets are dropped instantly at the switches because it is the default behavior. After starting the controller each host can reach the others and communicate with them.
	\end{adjustwidth}
	
	\section*{1.4.2 Flowtables?}
	\begin{adjustwidth}{2em}{2em}
		\begin{enumerate}
			\item \textsc{linear} \\
			\includegraphics[scale=0.42]{linear_flowtables.png}
			\item \textsc{single} \\
			\includegraphics[scale=0.42]{single_flowtables.png}
			\item \textsc{tree} \\
			\includegraphics[scale=0.42]{tree_flowtables_1.png} \\
			\includegraphics[scale=0.42]{tree_flowtables_2.png} \\
			\includegraphics[scale=0.42]{tree_flowtables_3.png} \\
		\end{enumerate}
		In the flowtables we can see that each packet has an in\_port which tells the switch from which port the packet came from. Each of the packets also has the information on where its destination is such that the output port can be computed. So when a packet from one port with a certain destination arrives the switch can lookup in the flowtable to determine which output-port the packet is sent out from.
	\end{adjustwidth}	
\end{document}