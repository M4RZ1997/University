\documentclass{report}
 \usepackage{hyperref}
 \usepackage[hoffset = 1px]{geometry}
 \usepackage{scalerel,amssymb}
 \usepackage{paralist}
 \begin{document}
 \begin{center}
 \huge{\textbf{\underline{Computer Vision}}}
 \end{center}
 \hfill \\ \\
 
 \underline{\textit{Important Information:}} \\ \\
 	course webpage: http://cvg.unibe.ch/teaching \\
 	textbooks: http://computervisionmodels.com \\
 				Vision Object Recognition \\
 				An invitation to 3-D vision: From images to geometric models \\
 	course requirements: 2 Assignments (60/100 points for pass) \\
 	Tutorials (weekly): aim is exam preparation \\
 	Final Mark: 70\% Exam and 30\% Assignments \\
 	First Assignment due November 12 (Image inpainting and blending) \\
 	Second Assignment due December 10 (3D reconstruction) \\
 	
 \chapter{The goal of computer vision}
 \begin{compactenum}[$\bullet$]
 	\item To extract "meaning" from pixels
 		\begin{enumerate}
 			\item semantic information
 			\item geometric information
 		\end{enumerate}
 	\item Vision is useful
 		\begin{enumerate}
 			\item automatization of large scale processing
 			\item much higher flow number (faster)
 		\end{enumerate}
 	\item Vision is interesting
 	\item \textbf{BEWARE:} Human vision is easy, but computer vision is difficult
 		\begin{enumerate}
 			\item Half of primate cerbral cortex is devoted to visual processing
 			\item Achieving human-level perception is probably "AI-complete"
 		\end{enumerate}
 \end{compactenum}
 
 \end{document}