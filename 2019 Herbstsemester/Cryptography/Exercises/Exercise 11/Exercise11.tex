\documentclass{report}
\usepackage[headheight=30pt, textheight=600pt]{geometry}
\usepackage{paralist}
\usepackage{scalerel,amssymb}
\usepackage{amsmath}
\usepackage{colortbl}
\usepackage{array}
\usepackage{multirow}

\usepackage{graphicx}
\usepackage{stix}

\newcommand{\tableflip}{$($\rotatebox{45}{$\smile$}$^{\circ}\smwhtsquare^{\circ})\rotatebox{45}{$\smile$}\mkern-6mu\frown$\raisebox{0.5ex}{$\bot$}$\mkern-3.5mu-\mkern-3.5mu$\raisebox{0.5ex}{$\bot$}}

\usepackage{stackengine}
\def\apeqA{\SavedStyle\sim}
\def\apeq{\setstackgap{L}{\dimexpr.5pt+1.5\LMpt}\ensurestackMath{%
  \ThisStyle{\mathrel{\Centerstack{{\apeqA} {\apeqA} {\apeqA}}}}}}

\usepackage{fancyhdr}
\fancyhead[L]{\LARGE Cryptography \\
\Large Exercise 11}
\fancyhead[R]{16-124-836 \\
Marcel \textsc{Zauder}}
\renewcommand{\headrulewidth}{0.4pt}
\fancyfoot[C]{\thepage}
\renewcommand{\footrulewidth}{0.4pt}

\usepackage{hyperref}

\begin{document}
	\pagestyle{fancy}
	
	\section*{11.1 Computing with encrypted messages}
	\subsection*{ElGamal}
	The encoding process looks as follows:
	\[
		Enc(pk, m) \ = \ (g^{r}, m \cdot Y^{r})
	\]
	For $m_1$ and $m_2$ we then get:
	\begin{align*}
		Enc(pk, m_1) \ & = \ (g^{r_1}, m_1 \cdot Y^{r_1}) \\
		Enc(pk, m_2) \ & = \ (g^{r_2}, m_2 \cdot Y^{r_2})
	\end{align*}
	For the operations $\otimes$ and $\oplus$ we then get:
	\begin{align*}
		Enc(pk, m_1) \ \otimes \ Enc(pk, m_2) \ & = \ (g^{r_1}, m_1 \cdot Y^{r_1}) \ \otimes \ (g^{r_2}, m_2 \cdot Y^{r_2}) \\
		& = \ (g^{r_1} \ \otimes \ g^{r_2}, m_1 \cdot Y^{r_1} \ \otimes \ m_2 \cdot Y^{r_2}) \\
		The \ \otimes \ can \ be \ replaced &\ with \ a \ multiplication \ (\cdot): \\
		& = \ (g^{r_1} \ \cdot \ g^{r_2}, m_1 \cdot Y^{r_1} \ \cdot \ m_2 \cdot Y^{r_2}) \\
		& = \ (g^{r_1 + r_2}, \underbrace{m_1 \ \cdot \ m_2}_{m_3} \cdot Y^{r_1 + r_2}) \\
		& = \ Enc(pk, m_3) & with \ m_3 \ = \ m_1 \cdot m_2
	\end{align*}
	\subsection*{RSA}
	The encoding process looks as follows:
	\[
		Enc(pk, m) \ := \ m^{pk} \ \% N
	\]
	For $m_1$ and $m_2$ we then get:
	\begin{align*}
		Enc(pk, m_1) \ & = \ m_1^{pk} \ \% N \\
		Enc(pk, m_2) \ & = \ m_2^{pk} \ \% N
	\end{align*}
	For the operations $\otimes$ and $\oplus$ we then get:
	\begin{align*}
		Enc(pk, m_1) \ \otimes \ Enc(pk, m_2) \ & = \ m_1^{pk} \ \% N \ \otimes \ m_2^{pk} \ \% N \\
		& = \ (m_1^{pk} \otimes m_2^{pk}) \ \% N \\
		The \ \otimes \ can \ be \ replaced &\ with \ a \ multiplication \ (\cdot): \\
		& = \ (m_1^{pk} \cdot m_2^{pk}) \ \% N \\
		& = \ ((\underbrace{m_1\cdot m_2}_{m_3})^{pk}) \ \% N \\
		& = \ Enc(pk, m_3) & with \ m_3 \ = \ m_1 \cdot m_2
	\end{align*}
	\newpage
	\section*{11.2 RSA parameters}
	\subsection*{11.2.a Why $e$ must be odd?}
	$e$ must be \textbf{odd}, because if $e$ would be \textbf{even} we can only reach \textbf{even} values ibn $\mathbb{G}$ (see Exercise 02).
	\subsection*{11.2.b Given $N$ and $\Phi(N)$}
	We have given:
	\begin{align*}
		N \ & = \ pq \\
		\Phi(N) \ & = \ (p-1) \ \cdot \ (q-1)
	\end{align*}
	Therefore we can compute:
	\begin{center}
		\begin{tabular}{lllllll}
			\begin{tabular}{|l|}
				$N \ = \ pq$ \\
				$\Phi(N) \ = \ (p-1) \ \cdot \ (q-1)$
			\end{tabular}
			&
			$\Leftrightarrow$
			&
			\begin{tabular}{|l|}
				$N \ = \ pq$ \\
				$q \ = \ \frac{\Phi(N)}{p-1} + 1$
			\end{tabular}
			&
			$\Rightarrow$
			&
			$N \ = \ p \cdot \left(\frac{\Phi(N)}{p-1} + 1\right)$
		\end{tabular}
	\end{center}
	With this we can now compute $p$:
	\begin{align*}
		& & N \ & = \ p \cdot \left(\frac{\Phi(N)}{p-1} + 1\right) \\
		& \Leftrightarrow & N \cdot (p-1) \ & = \ p \cdot \Phi(N) + p^2 - p \\
		& \Leftrightarrow & 0 \ & = \ p^2 + p \cdot ( \Phi(N) - N - 1 ) + N \\
		& \Leftrightarrow & p_{^1/_2} \ & = \ -\frac{\Phi(N) - N - 1}{2} \pm \sqrt{\left(\frac{\Phi(N) - N - 1}{2}\right)^2 - N}
	\end{align*}
	The solutions $p_1$ and $p_2$ are then the prime factorization of $N$.
	\section*{11.3 Bad choice of prime factors}
	\subsection*{11.3.a $p$ is "small"}
	\subsection*{11.3.a $\mid p - q \mid$ is "small"}
	\section*{11.4 RSA oracle}
	
\end{document}