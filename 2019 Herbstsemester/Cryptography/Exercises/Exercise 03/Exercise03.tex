\documentclass{report}
\usepackage[headheight=50pt,]{geometry}
\usepackage{paralist}
\usepackage{scalerel,amssymb}
\usepackage{amsmath}

\usepackage{fancyhdr}
\fancyhead[L]{\LARGE Cryptography \\
\Large Exercise 01}
\fancyhead[R]{16-124-836 \\
Marcel \textsc{Zauder}}
\renewcommand{\headrulewidth}{0.4pt}
\fancyfoot[C]{\thepage}
\renewcommand{\footrulewidth}{0.4pt}

\usepackage{hyperref}

\begin{document}
	\pagestyle{fancy}
	
	\section*{3.1 Nested Encription Scheme}
	\begin{enumerate}[]
		\item Prove that $\Sigma$ satisfies one-time secrecy, then so does $\Sigma ^2$:
		\begin{enumerate}[]
			\item The given library of function on the exercise sheet will be called $L_2$ of $\Sigma ^2$. \\
			We know that $\Sigma$ satisfies the one-time secrecy with $\Sigma$.Enc(k,m) = c, with library $L_1$. \\
			It is clear that $L_{OTS-L} \equiv L_{OTS-R}$, which means $Pr[A \diamond L_{OTS-L} \rightarrow 1] \ = \ Pr[A \diamond L_{OTS-R} \rightarrow 1]$, for any $A$. \\
			The scheme $\Sigma$ is used to encrypt $m_L$ and $m_R$ into $c_L$ and $c_R$ which are distributed equally. These encrypted ciphertexts are then encrypted again to get $c_{L2}$ and $c_{R2}$ which are still equally distributed. The used Library will be called $L'_1 \diamond L_1$. \\
			$L'_1 \diamond L_1$ will satisfy one-time secrecy because $L_{OTS-L} \equiv L_{OTS-R}$ with an appropriate Eavesdrop($m_L$, $m_R$). \\
			Because in $L'_1 \diamond L_1$ and $L_2$ the same is done, the produced ciphertexts are distributed the same (equally) and therefore one-time secrecy is given in $L_2$.
		\end{enumerate}
	\end{enumerate}
	\hfill \\
	\section*{3.2 Negligible Functions}
	\subsection*{3.2.a}
	\begin{enumerate}
		\item $\frac{1}{2^{\frac{\lambda}{2}}}$ Negligible?: \\
		It is negligible because: $\lim_{\lambda \rightarrow \infty} (p(\lambda) \cdot \frac{1}{2^{\frac{\lambda}{2}}}) \ = \ 0$
		\item $\frac{1}{\lambda^2}$ Negligible?: \\
		No, it is not negligible because for $p(\lambda) = \lambda^2$ we have: \\
		$\lim_{\lambda \rightarrow \infty} (p(\lambda) \cdot \frac{1}{\lambda^2})\ = \ \lim_{\lambda \rightarrow \infty} (\lambda^2 \cdot \frac{1}{\lambda^2}) \ = \ 1$
		\item $\frac{1}{\lambda^{\frac{1}{\lambda}}}$ Negligible?: \\
		No, it is not because: $\lim_{\lambda \rightarrow \infty} (p(\lambda) \cdot \frac{1}{\lambda^{\frac{1}{\lambda}}}) \ = \ \lim_{\lambda \rightarrow \infty} (p(\lambda) \cdot \frac{1}{\underbrace{\lambda^{\frac{1}{\lambda}}}_1}) \ = \ \lim_{\lambda \rightarrow \infty} (p(\lambda)) \neq 0 \ \forall p(\lambda)$
		\item $\frac{1}{\sqrt{\lambda}}$ Negligible?: \\
		No, it is not negligible because for $p(\lambda) = \sqrt{\lambda}$ we have: \\
		$\lim_{\lambda \rightarrow \infty} (p(\lambda) \cdot \frac{1}{\sqrt{\lambda}})\ = \ \lim_{\lambda \rightarrow \infty} (\sqrt{\lambda} \cdot \frac{1}{\sqrt{\lambda}}) \ = \ 1$
		\item $\frac{1}{2^{\sqrt{\lambda}}}$ Negligible?: \\
		It is negligible because: $\lim_{\lambda \rightarrow \infty} (p(\lambda) \cdot \frac{1}{2^{\sqrt{\lambda}}}) \ = \ 0$
	\end{enumerate}
	\subsection*{3.2.b}
	\begin{enumerate}[\textbullet]
		\item f(), g() are negligible $\Rightarrow$ f() $\cdot$ g() is negligible? \\
		We know:
		\begin{align*}
			\lim_{\lambda \rightarrow \infty} (p(\lambda) \cdot f(\lambda)) \ & = \ 0 \\
			\lim_{\lambda \rightarrow \infty} (p(\lambda) \cdot g(\lambda)) \ & = \ 0 \\
			\Rightarrow \ \lim_{\lambda \rightarrow \infty} (p(\lambda) \cdot (f(\lambda) \cdot g(\lambda))) \ & = \ \lim_{\lambda \rightarrow \infty} ((p(\lambda) \cdot f(\lambda)) \cdot g(\lambda)) \\
			& = \ \underbrace{\lim_{\lambda \rightarrow \infty} (p(\lambda) \cdot f(\lambda))}_0 \cdot \underbrace{\lim_{\lambda \rightarrow \infty} (g(\lambda))}_0 \\
			& = 0 & \square
		\end{align*}
		\item Example s.t. f() and g() are negligible but $\frac{f()}{g()}$ is not: \\
		If $f()=g()=\frac{1}{2^{\lambda}}$, both are clearly negligible, but $\frac{f()}{g()}= \frac{\frac{1}{2^{\lambda}}}{\frac{1}{2^{\lambda}}} = \frac{2^{\lambda}}{2^{\lambda}} = 1$ is clearly not.
	\end{enumerate}
	\section*{3.3 Hashrate}
	\subsection*{3.3.a CPU with 2GHz}
	\begin{enumerate}[]
		\item \textit{Assuming you have one Intel CPU with 2GHz clock speed, how many cycles per block can one have in case of a single-threaded AVX1 implementation? How much is the hash rate?}
		\begin{enumerate}[\textbullet]
			\item 1Mb = 1'000'000 bytes
			\item 2Ghz = 1 * $10^9$ Hz (cycles/sec)
			\item From the given paper we can assume that the performance of SHA-256 will be most likely be constant at 12.8 cycles/byte
		\end{enumerate}
		\item Therefore we have:
		\begin{align*}
			12.8 \ \frac{cycles}{byte} \times 1'000'000 \textit{ bytes } \ & = \ 12'800'000 \textit{ cycles} \\
			12'800'000 \textit{ cycles } \div \ 2'000'000'000 \ \frac{cycles}{sec} \ & = \ 0.0064 \textit{ sec} \\
			1 \textit{ sec } \div  \ 0.0064 \textit{ sec } \ & = \ 156.25 \textit{ hashes per second}
		\end{align*}
	\end{enumerate}
	\subsection*{3.3.b Bitcoin}
		\begin{enumerate}[]
			\item Current hashrate is 93'241'227 * $10^{12}$ hashes per second (3.10.2019 2:00)
			\begin{align*}
				93'241'227 \ * \ 10^{12} \ \div \ 156.25 \ & \approx \ 6 \ * 10^{17}
			\end{align*}
			\item So $\sim 6 \ * 10^{17}$ such CPUs are needed to compute the current hash rate of bitcoin.
		\end{enumerate}
	\section*{3.4 A Random Cipher}
	\subsection*{3.4.a Description}
	\begin{enumerate}[]
		\item $\Sigma$.M = $\Sigma$.C = $\{0,1\}^{\kappa}$ \\
		$\Sigma$.K = $\{0,1\}^?$
	\end{enumerate}
	\begin{tabular}{lllll}
		\begin{tabular}{l}
			$\Sigma$.KeyGen() = k $\leftarrow$ $\{0,1\}^?$
		\end{tabular}
		&
		,
		&
		\begin{tabular}{l}
			\underline{$\Sigma$.Enc(k,m)} \\
			c = ??? \\
			return c
		\end{tabular}
		&
		,
		&
		\begin{tabular}{l}
			\underline{$\Sigma$.Dec(k,c)} \\
			m = ??? \\
			return m
		\end{tabular}
	\end{tabular}
	\subsection*{3.4.b Upper Bound}
	\begin{enumerate}[]
		\item The chance to guess m randomly out of c is:
		\begin{align*}
			P[A(c) \Rightarrow m] \ & = \ 1 -(1-\frac{q}{2^k}) \textit{ invers of guessing q-times false.} \\
			& = \ \frac{q}{2^k}
		\end{align*}
		\item For $q \ \rightarrow \ 2^k$ the probability gets to 1.
	\end{enumerate}
\end{document}