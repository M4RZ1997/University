\documentclass{report}
\usepackage[headheight=30pt, textheight=650pt]{geometry}
\usepackage{paralist}
\usepackage{scalerel,amssymb}
\usepackage{amsmath}
\usepackage{colortbl}
\usepackage{array}
\usepackage{multirow}

\usepackage{graphicx}
\usepackage{stix}

\newcommand{\tableflip}{$($\rotatebox{45}{$\smile$}$^{\circ}\smwhtsquare^{\circ})\rotatebox{45}{$\smile$}\mkern-6mu\frown$\raisebox{0.5ex}{$\bot$}$\mkern-3.5mu-\mkern-3.5mu$\raisebox{0.5ex}{$\bot$}}

\usepackage{stackengine}
\def\apeqA{\SavedStyle\sim}
\def\apeq{\setstackgap{L}{\dimexpr.5pt+1.5\LMpt}\ensurestackMath{%
  \ThisStyle{\mathrel{\Centerstack{{\apeqA} {\apeqA} {\apeqA}}}}}}

\usepackage{fancyhdr}
\fancyhead[L]{\LARGE Cryptography \\
\Large Exercise 09}
\fancyhead[R]{16-124-836 \\
Marcel \textsc{Zauder}}
\renewcommand{\headrulewidth}{0.4pt}
\fancyfoot[C]{\thepage}
\renewcommand{\footrulewidth}{0.4pt}

\usepackage{hyperref}

\begin{document}
	\pagestyle{fancy}
	
	\section*{9.1 Diffie-Hellman assumptions}
		\subsection*{9.1.a Bad Event Lemma on CDH}
			\begin{tabular}{lllll}
				\begin{tabular}{|l|}
					\hline
					$a, b \leftarrow \mathbb{Z}_q$ \\
					\\
					\textsc{Seed}(): \\
			  		\ \ \textbf{return} ($g^a, g^b$) \\
					\\
					\textsc{Check}(x): \\
			  		\ \ if $x = {(g^a)}^b$: \\
			    	\ \ \ \ \textbf{return} 1 \\
			  		\ \ else: \\
			    	\ \ \ \ \textbf{return} 0 \\
			    	\hline
				\end{tabular}
				&
				&
				\begin{tabular}{|l|}
					\hline
					$a, b \leftarrow \mathbb{Z}_q$ \\
					\\
					\textsc{Seed}(): \\
			  		\ \ \textbf{return} ($g^a, g^b$) \\
					\\
					\textsc{Check}(x): \\
			  		\ \ \textbf{return} 0 \\
			  		\hline
				\end{tabular}
				&
				&
				\begin{tabular}{p{7.5cm}}
					Because we know that for any given $(g^a, g^b)$ an attacker cannot compute $g^{ab}$ in polynomial time with non negligible probability, the advantage of any algorithm is also non-negligible. Therefore the two libraries are indistinguishable.
				\end{tabular}
			\end{tabular}
			\\
			
		\subsection*{9.1.b Relation between DDH and CDH? What is the problem if one can solve CDH?}
			CDH is a "stronger" assumption than DDH, because one need to compute $g^{ab}$ in non-negligible time, so therefore if CDH is solveable and non-negligible, the same can be said about DDH, so that it is not indistinguishable anymore. Therefore if one can solve the CDH in non-negligible time so is the discrete logarithm problem.
	
	\section*{9.2 Man-in-the-middle attack}
		As \textsc{Alice} will receive the wrong value $g^{b'}$ which was changed by \textsc{Eve} she will raise this to her chosen number $a$ which will result in the key $(g^{b'})^{a} \ = \ g^{b' \cdot a}$. The same is happening to \textsc{Bob} who will calculate the key $(g^{a'})^{b} \ = \ g^{a' \cdot b}$. \\
		Because the resulting keys are not equal, \textsc{Alice} and \textsc{Bob} will not be able to decrypt the messages sent from one to the other (, \textit{without any interception in between them}). \\
		\textsc{Eve} can intercept the messages of \textsc{Alice} and \textsc{Bob}. Because she/he can compute the keys which \textsc{Alice} and \textsc{Bob} will have, by raising the intercepted $g^a$ and $g^b$ with either $b'$ or $a'$, she/he can decrypt the messages and encrypt them with the keys corresponding to \textsc{Alice} or \textsc{Bob}. Therefore \textsc{Alice} and \textsc{Bob} will not notice that anything is wrong, but \textsc{Eve} can read the sent messages.
	\section*{9.3 Quadratic residues}
		We know that $g$ is a \textit{primitive root}, therefore it is:
		\[
			\langle g \rangle _p \ = \ \mathbb{Z}_p^{\star}
		\]
		We want to show that $g^a \in \mathbb{Q}\mathbb{R}_p^{\star}$ iff $a$ is even. Suppose that this is true:
		\begin{align*}
			& & g^a \in \mathbb{Q}\mathbb{R}_p^{\star}
			& & \Rightarrow & & g^{a \cdot \frac{p-1}{2}} \ \equiv _p \ 1
			& & \Leftrightarrow & & g^{\frac{a}{2} \cdot (p-1)} \ \equiv_p \ 1
		\end{align*}
		Because $a$ is even we can substitute $\frac{a}{2}$ with $b$ which is still in $\mathbb{Z}$. Therefore we have:
		\begin{align*}
			& g^{b \cdot (p-1)} \ \equiv_p \ 1 & (1)
		\end{align*}
		Because $g$ is a \textit{primitive root} of $\mathbb{Z}_p^{\star}$:
		\[
			g^a \ mod \ p \ = \ g^{a + b \cdot (p-1)} \ mod \ p
		\]
		Because $g^0 \equiv_p 1$, equation (1) is shown and therefore $g^a \in \mathbb{Q}\mathbb{R}_p^{\star}$.
	
\end{document}