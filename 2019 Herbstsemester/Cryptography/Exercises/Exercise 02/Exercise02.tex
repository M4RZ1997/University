\documentclass{report}
\usepackage[headheight=50pt,]{geometry}
\usepackage{paralist}
\usepackage{scalerel,amssymb}
\usepackage{amsmath}

\usepackage{fancyhdr}
\fancyhead[L]{\LARGE Cryptography \\
\Large Exercise 01}
\fancyhead[R]{16-124-836 \\
Marcel \textsc{Zauder}}
\renewcommand{\headrulewidth}{0.4pt}
\fancyfoot[C]{\thepage}
\renewcommand{\footrulewidth}{0.4pt}

\usepackage{hyperref}

\begin{document}
	\pagestyle{fancy}
	
	\section*{2.1 Basics on libraries}
	\subsection*{2.1.1 Probabilities}
	$\mathbb{Z}_6 = \{0,1,2,3,4,5\}$ \\
	\begin{compactenum}[$\bullet$]
		\item \textbf{Pr[$A_1 \diamond L_1 \Rightarrow 1$]:}
		\begin{enumerate}[]
			\item Pr[$(r_1 \leftarrow \mathbb{Z}_6)$ $\stackrel{?}{=}$ $(r_2 \leftarrow \mathbb{Z}_6)$] = $\frac{1}{6}$
		\end{enumerate}	
		\item \textbf{Pr[$A_1 \diamond L_2 \Rightarrow 1$]:}
		\begin{enumerate}[]
			\item Pr[0 $\stackrel{?}{=}$ 0] = 1
		\end{enumerate}				
		\item \textbf{Pr[$A_2 \diamond L_1 \Rightarrow 1$]:}
		\begin{enumerate}[]
			\item Pr[$(r \leftarrow \mathbb{Z}_6)$ $\stackrel{?}{\geq}$ 3] = $\frac{1}{2}$
		\end{enumerate}	
		\item \textbf{Pr[$A_2 \diamond L_2 \Rightarrow 1$]:}
		\begin{enumerate}[]
			\item Pr[0 $\stackrel{?}{\geq}$ 3] = 0
		\end{enumerate}	
	\end{compactenum}
	\subsection*{2.1.2 Equivalent libraries}
	Two Libraries $L_{left}$ and $L_{right}$ are equivalent iff:
	\[
		P[A\diamond L_{left} \rightarrow 1] = P[A\diamond L_{right} \rightarrow 1]
	\]
	\begin{compactenum}[$\bullet$]
		\item 	\begin{tabular}{lll}
					\begin{tabular}{|c|}
						\hline \textbf{$L_{left}$} \\
						\hline \multicolumn{1}{|l|}{\underline{\textsc{Query():}}} \\
						x $\leftarrow$ $\{0,1\}^n$ \\
						return x \\
						\hfill \\
						\hline
					\end{tabular}
					&
					$\stackrel{?}{\equiv}$
					&
					\begin{tabular}{|c|}
						\hline \textbf{$L_{right}$} \\
						\hline \multicolumn{1}{|l|}{\underline{\textsc{Query():}}} \\
						x $\leftarrow$ $\{0,1\}^n$ \\
						y := $\overline{x}$ \\
						return y \\
						\hline
					\end{tabular}
				\end{tabular}
			\begin{enumerate}[]
				\item Because we can make a 1:1 correspondence (we could make a bijection) for each return value of $L_{left}$ to each return value of $L_{right}$ the probabilities for each return value are equal and therefore the libraries are equivalent. \\
			\end{enumerate}
		\item 	\begin{tabular}{lll}
					\begin{tabular}{|c|}
						\hline \textbf{$L_{left}$} \\
						\hline \multicolumn{1}{|l|}{\underline{\textsc{Query():}}} \\
						x $\leftarrow$ $\mathbb{Z}_n$ \\
						return x \\
						\hfill \\
						\hline
					\end{tabular}
					&
					$\stackrel{?}{\equiv}$
					&
					\begin{tabular}{|c|}
						\hline \textbf{$L_{right}$} \\
						\hline \multicolumn{1}{|l|}{\underline{\textsc{Query():}}} \\
						x $\leftarrow$ $\mathbb{Z}_n$ \\
						y := 2x \% n \\
						return y \\
						\hline
					\end{tabular}
				\end{tabular}
			\begin{enumerate}[]
				\item \textbf{For "even" n's:} \\
				Let us assume that n = 2 and we calculate the probability of the return value being 1. In this case $\mathbb{Z}_2$ = $\{0,1\}$ and the probability of $L_{left}$ returning 1 is therefore $\frac{1}{2}$. The library $L_{right}$ will only return 1 if there is a possibility to solve the equation 1 = 2x $\%$ 2, with x $\in$ $\mathbb{Z}_2$. Because there is no possible result $L_{right}$ cannot return 1, so the probability is 0 and the libraries are therfore not equivalent. \\
				\item \textbf{For "uneven" n's:} \\
				For uneven n, the distributions are:
				\[
					\mathbb{Z}_n  = \{0,1,2,...,n-1\} $$ $$
					2 \cdot \mathbb{Z}_n \% n = \{0,2,4,...,n-1,1,3,..., n-2\}
				\]
				Therefore the second distribution is a permutation of the first one and therefore the libraries are equivalent. \\
			\end{enumerate}
		\item 	\begin{tabular}{lll}
					\begin{tabular}{|c|}
						\hline \textbf{$L_{left}$} \\
						\hline \multicolumn{1}{|l|}{\underline{\textsc{Query():}}} \\
						x $\leftarrow$ $\{0,1\}^n$ \\
						y $\leftarrow$ $\{0,1\}^n$ \\
						return x \& y \\
						\hline
					\end{tabular}
					&
					$\stackrel{?}{\equiv}$
					&
					\begin{tabular}{|c|}
						\hline \textbf{$L_{right}$} \\
						\hline \multicolumn{1}{|l|}{\underline{\textsc{Query():}}} \\
						z $\leftarrow$ $\{0,1\}^n$ \\
						return z \\
						\hfill \\
						\hline
					\end{tabular}
				\end{tabular}
			\begin{enumerate}[]
				\item Let us assume that n = 1. The probability of $L_{left}$ returning 0 is $\frac{3}{4}$ because this is returned if $((x=0) \wedge (y=0)) \vee ((x=0) \wedge (y=1)) \vee ((x=1) \wedge (y=0))$. Only if $((x=1) \wedge (y=1))$ $L_{left}$ returns 1. $L_{right}$  will return 0 with a possibility of $\frac{1}{2}$. Therefore they are not equivalent.
			\end{enumerate}
	\end{compactenum}
	\hfill \\
	
	\section*{2.2 Security of a modified One-time Pad (OTP)}7
	Given the two libraries from the lecture, we need to show that $L_{OTS_left} \equiv L_{OTS_right}$ so we can conclude the one-time secrecy: \\
	For two arbitrary messages $m_1$ and $m_2$, the eavesdrop() function will return the following bit string for either of the two libraries:
	\[
		c_1 c_2 \cdots c_{n-2} c_{n-1} c_n $$ $$
		\textit{whereas: } c_{n - 1} = c_n = 0 \textit{ and } c_1, c_2, \cdots, c_{n-2} \textit{ are uniformaly distributed in } \{0,1\}^{n-2} 
	\]
	Because in either cases the ciphertexts will be distributed in the same way, a distinguishable algorithm A still will be unable to differ between those two libraries. This implies that both libraries are exchangeable due to the fact that $P[A \diamond L_{OTS-left} \Rightarrow 1] = P[A \diamond L_{OTS-right} \Rightarrow 1]$. So this cipher will provide a one time secrecy. $\square$
	\newpage
	\section*{2.3 Construction of a distinguisher}
	First we can make a few assumptions:
	\begin{enumerate}[\hspace{1cm} 1.]
		\item If at least one of $m$ \textsc{or} $k$ is even the result for $(k \times m) \% 10$ is even
		\item Only if $m$ and $k$ are odd the result for $(k \times m) \% 10$ is odd
	\end{enumerate}
	Therefore we can compute the following table: \\
	\begin{center}	
		\begin{tabular}{ccc}
			\hline
	    			 & $k$ even & $k$ odd \\
			\hline
			$m$ even & $c$ even & $c$ even \\
			$m$ odd  & $c$ even & $c$ odd \\
			\hline
		\end{tabular}
	\end{center}
	Now we can define a distinguishing algorithm A:
	\begin{align*}
		A: \ & \\
		& \ m_l \leftarrow 2 \\
		& \ m_r \leftarrow 3 \\
		& \ c = Eavesdrop(m_l, m_r) \\
		& \ if \ (c \textit{ mod } 2 \ == \ 0) \ \{ \\
		& \ \ \ \ \frac{2}{3} \textit{ likelihood that } m_l \textit{ encrypted} \\
		& \ \ \ \ return \ 0 \\
		& \ \} \\
		& \ else \ \{ \\
		& \ \ \ \ \textit{Guaranteed that } m_r \textit{ is encrypted} \\
		& \ \ \ \ return \ 1 \\
		& \ \} \\
		& \ end
	\end{align*}
	Therefore for the probability follows:
	\[
		P[A \diamond L_{OTS-left} \Rightarrow 1] = 0 \neq \frac{1}{2} = P[A \diamond L_{OTS-right} \Rightarrow 1]
	\]
	We can see that this leads to the conclusion that the two libraries are \textsc{not} exchangeable and therefore the one-time secrecy cannot be provided.
	\hfill \\
	
	\section*{2.4* Size of the OTP key space}
	-
	
\end{document}