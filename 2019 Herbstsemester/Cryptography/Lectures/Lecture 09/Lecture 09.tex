\documentclass{report}
\usepackage{hyperref}
\usepackage{geometry}
\usepackage{scalerel,amssymb}
\usepackage{paralist}
\usepackage{colortbl}
\usepackage{amsmath}
\usepackage{pgfplots}
\usepackage{xcolor}
\usepackage{tikz}
\usetikzlibrary{positioning}
\usetikzlibrary{shapes.geometric, arrows}
\tikzstyle{arrow} = [thick,->,>=stealth]

\usepackage{stackengine}
\def\apeqA{\SavedStyle\sim}
\def\apeq{\setstackgap{L}{\dimexpr.5pt+1.5\LMpt}\ensurestackMath{%
  \ThisStyle{\mathrel{\Centerstack{{\apeqA} {\apeqA} {\apeqA}}}}}}

\begin{document}
\begin{center}
\huge{\textbf{\underline{Cryptography}}}
\end{center}

	{\let\clearpage\relax \chapter*{Encryption modes}}
	\begin{tabular}{|l|c|c|c|c|}
		\hline
		& ECB & CBC & CTR & OFB \\
		\hline
		secure? & x & y & y & y \\
		encryption parallel? & & x & y & x \\
		decryption parallel? & & y & y & x\\
		random access decryption? & & y & y & x \\
		\hline
	\end{tabular} \\
	\subsection*{Length expansion}
	\underline{if} dec. error \underline{then} \\
	\textbf{return} \textsc{Error} \\ \\
	\underline{if} dec. OK \underline{then} \\
	\textbf{return} $*-*$
	\subsection*{Storage encryption}
	\begin{enumerate}[-]
		\item where to store the IV or nonce?
		\item because expansion not possible:\\
		Enc $B^{512} \rightarrow B^{512}$
		\item $\rightarrow$ use nonce derived from address \\
		ESSIV: where IV for sector i is \\
		$IV := Enc(H(k),i)$ \\
		encrypted salt sector IV
	\end{enumerate}
	{\let\clearpage\relax \chapter*{9 Diffie-Hellman Key Agreement}}
	\section*{Modular arithmetic}
	\begin{enumerate}[-]
		\item Integer divisions: For $a,d \in \mathbb{Z}$ there exist a unique \underline{quotient} $q$ and a unique \underline{remainder} $r$ s.t.:
		\begin{align*}
			a = d \cdot q + r & & and \ \ 0 \leq r \leq \mid d-1 \mid
		\end{align*}
		\item Since $q$ and $r$ are unique:
		\begin{align*}
			q \ & = \ a \ \underline{div} \ d \ & = \ \lfloor \frac{a}{d} \rfloor \\
			r \ & = \ a \ \underline{mod} \ d \ & = \ a \% d
		\end{align*}
		\item Relation "divides": $a \mid d$
	\end{enumerate}
	\subsection*{Congruence relation}
		$a \equiv b$ (mod $m$) or $a \equiv _m b$ \underline{if} $m \mid (a-b)$
		\textbf{"Integers mod m"}: $\mathbb{Z}_m \stackrel{def.}{=}\{0,1,...,m-1\}$ \\
		\textbf{\textsc{Note}}: $\underbrace{a \equiv _m b}_{equivalence \ relation} \ \neq \ ( \underbrace{a \ mod \ m \ = \ b \ mod \ m}_{euquality \ over \ \mathbb{Z}})$ \\
		\textbf{\textsc{Rules}}: $(a+b) \ mod \ m \ = \ (((a \ mod \ m) + (b \ mod \ m)) \ mod \ m)$
	\subsection*{Cyclic groups}
	\subsubsection*{Definition}
	A group $\langle F, \cdot , 1 \rangle$ consists of a \underline{set} $G$, an \underline{operation} $\cdot$ , and a \underline{neutral element} 1
	\begin{enumerate}
		\item $\forall a,b \in G: (a \cdot b) \in G$
		\item $\forall a: 1 \cdot a = a \cdot 1 = a$
		\item $\forall a \in G, \exists a^{-1} \in G: a \cdot a^{-1} = 1$
		\item associative
	\end{enumerate}
	\subsubsection*{Example}
	\begin{enumerate}
		\item $\mathbb{Z}_m \stackrel{def}{=} \langle \mathbb{Z}_m, +, 0 \rangle$
		\item $\mathbb{Z}_p^{\star} \stackrel{def}{=} \langle \{ 1,2,...,p-1 \} , \cdot, 1 \rangle$
	\end{enumerate}
	\subsubsection*{Definition}
	$\mid G \mid$ denotes the number of elements in $G$
	\subsubsection*{Definition}
	A finite group $G$ is \underline{cyclic} if some $g$ called \underline{generator} exists s.t. $G = \{ g^0, g^1, g^2, ... g^{\mid G \mid - 1}\}$ \\
	\underline{Notation}: $\langle g \rangle = G$ \\
	\underline{Integers mod m:} $\langle g \rangle _m \subset \mathbb{Z}_m^{\star}$
	\subsubsection*{Definition}
	If $\langle g \rangle _p = \mathbb{Z}_p^{\star}$, then $g$ is a \underline{primitive root}.
	\subsubsection*{Example}
	$\mathbb{Z}_{11}^{\star}$ \\
	$\langle 1 \rangle _{11} = \{ 1 \}$ \\
	$\langle 2 \rangle _{11} = \{ 1, 2, 4, 8, 5, 10, 9, 7, 3, 6 \}$ \\
	$\langle 2 \rangle _{11} = \{ 1, 3, 9, 5, 4 \}$ \\
	\subsubsection*{Definition}
	\begin{enumerate}[\textbullet]
		\item The number of elements in $G$ is also called the \underline{order} of $G$
		\item The order of $a \in G$ is the smallest i s.t. $a^i = 1$ (in $G$) \\
		$[$smallest i s.t. $g^i \equiv _m 1]$
	\end{enumerate}
	\subsubsection*{Lemma}
	For all primitive roots $g$:
	\[
		g^a = g^b \Leftrightarrow a \equiv _{\mid G \mid} b
	\]
	\subsubsection*{Example}
	\begin{enumerate}[\textbullet]
		\item $\mathbb{Z}_{p}^{\star}, \ p \ prime: \mid \mathbb{Z}_{p}^{\star} \mid = p-1$
		\item For $q \mid (p-1)$, and $q$ prime, there is a cyclic group of order $q$ ($q$ prime!), defined by multiplication modulo $p$ \\
		(think of: $\underbrace{p}_{safe \ prime} \ = \ 2 \cdot \underbrace{q}_{Sophie-Germain \ prime} + 1$ \\
		$p \ = \ m\cdot q + 1$, where $\mid p \mid \ = \ 2000$, but $q \ \approx \ 256$
	\end{enumerate}
	\subsection*{Discrete Logarithms}
	\subsubsection*{Definition}
	In a cyclic group $G$, the \underline{discrete logarithm} of $y \in G$ w.r.t a primitive root $g$ is $x \in \mathbb{Z}_{\mid G \mid}$ s.t. $g^x = y$.
	\subsubsection*{Definition}
	\textbf{\underline{Discrete Logarithm Problem (DLP):}} \\
	Given $y \leftarrow G$, compute $x$ s.t. $g^x = y$ \\ \\
	Group where the DLP is computationally hard:
	\begin{enumerate}[-]
		\item Prime-order subgroups of $\mathbb{Z}_{p}^{\star}$, $p$ prime (DSA, DH)
		\item groups defined over points on alliptic curves (ECDSA...) $[ \mid G \mid \geq 2^{256}]$
	\end{enumerate}
	
	\section*{Diffie-Hellman Key Agreement}
	\subsection*{Goal}
	Agree on a pseudorandom key.
	\begin{tabular}{lll}
		\begin{tabular}{l}
			\underline{Alice} \\
			$a \leftarrow \mathbb{Z}_{q}$ \\
			$X := g^a$ \\
			$Z_A := Y^a$
		\end{tabular}
		&
		\begin{tabular}{c}
			$G = \langle g \rangle$	 \\
			order $q$ \\
			\\
			$\stackrel{x}{\longrightarrow}$ \\
			$\stackrel{y}{\longleftarrow}$ \\
			Eve
		\end{tabular}
		&
		\begin{tabular}{l}
			\underline{Bob} \\
			$b \leftarrow \mathbb{Z}_{q}$ \\
			$Y := g^b$ \\
			$Z_B := X^b$
		\end{tabular}	
	\end{tabular} \\ \\
	\underline{\textsc{Claim:}} $ Z_A = Z_B$: $Z_Y = Y^a = (g^b)^a = (g^a)^b = X^b = Z_B$
	\subsection*{Security?}
	\begin{enumerate}[-]
		\item If Eve can compute DLOG, then not secure
		\item Want that $Z$ is pseudorandom
	\end{enumerate}
	\subsubsection*{Definition}
	Protocol $\Pi$ generates a key $k$ in $\Pi .K$ and a transcript $T \in \{ 0,1 \} ^{\star}$. \\
	$(T, k) \ \leftarrow \ EXEC(\Pi)$ \\
	$K.A$ protocol $\Pi$ is called \underline{secure} if: \\
	\begin{tabular}{lll}
		\begin{tabular}{l}
			\underline{$L^{\Pi}_{ka-real}$} \\
			\textsc{\underline{Query():}} \\
			\ $(T, k) \ \leftarrow \ EXEC(\Pi)$ \\
			\ \textbf{return} (T,k)
		\end{tabular}
		&
		$\apeq$
		&
		\begin{tabular}{l}
			\underline{$L^{\Pi}_{ka-rand}$} \\
			\textsc{\underline{Query():}} \\
			\ $(T, k) \ \leftarrow \ EXEC(\Pi)$ \\
			\ $k^{\star} \leftarrow \Pi .K$ \\
			\ \textbf{return} (T,$k^{\star}$)
		\end{tabular}
	\end{tabular}
\end{document}
 