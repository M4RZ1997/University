\documentclass{report}
 \usepackage{hyperref}
 \usepackage[hoffset = 1px]{geometry}
 \usepackage{scalerel,amssymb}
 \usepackage{paralist}
 \begin{document}
 \begin{center}
 \huge{\textbf{\underline{Cryptography}}}
 \end{center}
 \hfill \\ \\
 
 \underline{\textit{Important Information:}} \\ \\
 Exam: January 2020\\
 Lectures: 14:15-17:00, 106 HG\\
 Exercises: 10 points each, required 70\% \\
 Trivia: NONE
 
 {\let\clearpage\relax \chapter{Introduction}}
 Difference between \textbf{Cryptology} and \textbf{Cryptography}: \\
 Cryptology is the combination of Cryptography and Cryptoanalysis \\
 Cryptology concerns the protection of informstion in an adverserial context. \\
 \subsection{Classical Goals of Cryptography (CIA)}
 \begin{compactenum}[$\bullet$]
 	\item Confidentiality
 	\item Integrity
 	\item Availability
 \end{compactenum}
 \subsection{Techniques}
 \begin{compactenum}[$\bullet$]
 	\item encryption
 	\item digital signatures
 	\item hash functions
 	\item MACs
 \end{compactenum}
 \subsection{Modern goals}
 \begin{compactenum}[$\bullet$]
 	\item Anonymity
 	\item Zero-knowledge proofs
 \end{compactenum}
 
 \newpage
 
 \section{A model for a cryptosystem (Shannon, 48)}
 \begin{center}
 	KeyGen() (outputs secret key k) \\
 	$\downarrow$k \hspace{3.8cm} $\downarrow$k\\
 	(Alice)	$\stackrel{m}{\rightarrow }$ [Enc] $\stackrel{c}{\longrightarrow }$ [Dec] $\stackrel{\stackrel{\wedge}{m}}{\rightarrow }$ (Bob) \\
 	$\downarrow$c \\
 	(Eve) \\
 \end{center}
 	plaintext m, ciphertext c, decoded text $\stackrel{\wedge}{m}$ \\
 \subsection{Key genereaton algorithm}
 KeyGen() $\rightarrow$ k
 \subsection{Encryption algorithm}
 Enc(k,m) $\rightarrow$ c
 \subsection{Decription algorithm}
 Dec(k,c) $\rightarrow$ $\stackrel{\wedge}{m}$
 \subsection{Goals}
 \begin{enumerate}
 	\item \textit{Completeness}: \\
 		Bob obtains the message from Alice: \\
 		m $\stackrel{?}{=}$ $\stackrel{\wedge}{m}$
 		\item \textit{Security}: \\
 		"Eve obtains no useful information about the message m"
 \end{enumerate}
 
 \section{Hystoric cryptography}
 \begin{compactenum}[$\bullet$]
 	\item \textit{Scytale} (ancient Greece, 3rd. cent. BC)
 	\item \textit{Ceasar cipher} (Roman empire, 50 BC) (shift cipher) 
 	\begin{compactenum}[$\bullet$]
 		\item KeyGen(): k $\stackrel{R}{\leftarrow}$ \{0,1,...,25\}
 		\item Enc(k,l): return l+k
 		\item Dec(k,c): return c-k
 	\end{compactenum}
 	\item \textit{Monoalphabetic substitution} (uniform distributed permutation of the input)
 	\begin{compactenum}[$\bullet$]
 		\item Enc(k,l): return K[l]
 		\item Dec(k,c): return K$^{-1}$[c]
 	\end{compactenum}
 \end{compactenum}
 
 \chapter{Mathematical preliminories} 
 \section{Probability theory}
 \begin{compactenum}[$\star$]
 	\item A probability distribution P[ ] over a sample space $\Omega$ assigns a value in [0,1] to each $S \subseteq \Omega $ ("events") such that:
 	\begin{enumerate}
 		\item P[$\Omega$] = 1
 		\item P[A $\cup$ B] = P[A] + P[B] \\
 		for any $A,B\subseteq \Omega$ s.t. $A \cap B \neq \emptyset$
 	\end{enumerate}
 	\hfill \\
 	\item Random variable $X \in \chi$ ($\chi$-alphabet) is defined by an alphabet $\chi$ and a prob. distribution $P_X$ over $\chi$, s.t.:
 	\begin{center}
 		$P_X (x) = P[X=x]$ s.t. $\sum_{x\in \chi} P_X(x) = 1$
 	\end{center}
 	\hfill \\
 	\item A uniform random variable U over an alphabet $\mho$ satisfies:
 	\begin{center}
 		$P_U(u) = 1/\mid \mho \mid$ for $u \in \mho$
 	\end{center}
 	\underline{Example:} \\
 	P[Adversary A outputs a value k] $\leq 2^{-\lambda}, \lambda \in \{0,1\}^{\lambda}$
 \end{compactenum}
 \hfill \\ \\
 \textbf{Warning!!!!} "random" $\neq$ arbitrary \\
 
 \newpage
 
 \section{Notation}
 \begin{compactenum}[$\star$]
 	\item \textbf{$\leftarrow$ or $\stackrel{R}{\leftarrow}$} \\
 	For set S the notation: 
 	\begin{center}
 		$x \leftarrow S$
 	\end{center}
 	denotes that the value x ios chosen uniformly at random from set S
 	\begin{center}
 		$\forall s \in S: P[x \leftarrow S: x=s] = 1/\mid S \mid$
 	\end{center}
 	For a randomized algorithm R(y):
 	\begin{center}
 		$x \leftarrow R(y)$
 	\end{center}
 	denotes the experiment of running R on input y and assigning its output to x
 	\item \textbf{:=} (assignment operator)
 	\begin{center}
 		x := 1
 	\end{center}
 	\item \textbf{$\stackrel{?}{=}$} \\
 	\begin{center}
 		\underline{if} $x \stackrel{?}{=}$ \underline{then} ...
 	\end{center}
 \end{compactenum}
 
 \section{Kerckhoff's principle}
 	Design cryptosystem s.t. its security does not rely on the secrecy of the algrithm itself
 
 \subsection{One-time-pad (Vernam's cipher)}
 	\begin{compactenum}[$\bullet$]
 		\item Vernam ($\sim$ 1916)
 		\item Security proof by Shannon (1948)
 	\end{compactenum}
 	\underline{Syntax:} keys, messages are $\lambda$-bit strings $\Sigma$ = \{0,1\} $m \in \Sigma ^{\lambda} $ \\
 	\underline{KeyGen():} $k \leftarrow \Sigma ^{\lambda}$ \\
 	Enc(k,m): return $k \oplus m$ \\
 	Dec(k,c): return $k \oplus c$
 	
 \subsection{Completeness}
 \subsubsection{Theorem}
 \[
 	\forall m \in \Sigma ^{\lambda}, \forall k \in \Sigma ^{\lambda}, $$ $$
 	Dec(k, Enc(k,m)) = m
 \]
 \subsubsection{Proof}
 \[
 	k \oplus Enc(k,m) $$ $$
 	= k \oplus (k \oplus m) $$ $$
 	= (k \oplus k) \oplus m $$ $$
 	= 0^{\lambda} \oplus m $$ $$
 	= m
 \]
 
 \subsection{Security}
 Consider experiment that produces exactly the distribution seen by Eve. \\
 \underline{Eavesdrop(m)} \\
 \[
 	k \leftarrow \Sigma ^{\lambda} $$ $$
 	c \leftarrow m \oplus k $$ $$
 	\textit{return c}
 \]
 \subsubsection{Theorem}
 \[
 	\forall m \in \Sigma ^{\lambda}, \textit{ Eavesdrop(m) is a uniform random variable over } \Sigma ^{\lambda} $$ $$
 	\Rightarrow m \neq m': \textit{ Eavesdrop(m) has same distribution as Eavesdrop(m')}
 \]
 \subsubsection{Proof}
 \[
 	\forall m \in \Sigma ^{\lambda} $$ $$
 	\forall c \in \Sigma ^{\lambda} $$ $$
 	P[\textit{Eavesdrop(m)=c}] = ? $$ $$
 	\textit{\underline{We know:} } c = m \oplus k \Leftrightarrow k = m \oplus c $$ $$
 	P[\textit{Eavesdrop(m)=c}] = P[m \oplus k = c] = P[k = \underbrace{m \oplus c}_s ] = P[k=s] = 2^{- \lambda}
 \]
 
 \end{document}