\documentclass{article}
\usepackage[textheight = 600pt]{geometry}
\usepackage{paralist}
\usepackage[T1]{fontenc}
\usepackage{reledmac}
\usepackage{changepage}
\usepackage{amsmath}
\usepackage{scalerel,amssymb}

\usepackage{pgfplots}
\usepackage{tikz}
\usetikzlibrary{positioning}
\usetikzlibrary{shapes.geometric, arrows}
\tikzstyle{arrow} = [thick,->,>=stealth]

\usepackage{fancyhdr}
\fancyhead[L]{
	\begin{tabular}{l}
		\LARGE \textbf{\textsc{Distributed Algorithms}} \\
		\Large Exercise 10
	\end{tabular}
}
\fancyhead[R]{
	\begin{tabular}{r}
		16-124-836 \\
		Marcel \textsc{Zauder}
	\end{tabular}
}
\renewcommand{\headrulewidth}{0.4pt}
\fancyfoot[C]{\thepage}
\renewcommand{\footrulewidth}{0.4pt}

\usepackage{hyperref}

\begin{document}
	\pagestyle{fancy}
	\section*{10.1 Byzantine Randomized Consensus}
	\begin{adjustwidth}{2em}{2em}
		\subsection*{(a) Optimal Resilience for Byzantine Consensus?}
		\begin{adjustwidth}{2em}{2em}
			We assume that we have four processes $p_1$ to $p_4$, of which $p_1$ is the only faulty one. Process $p_2$ proposes the value $0$ but processes $p_3$ and $p_4$ are proposing $1$. At last we assume that the propose of any process to itself does not affect the decision process because it is delayed. \\
			In the first round process $p_3$ has received from $p_1$ and $p_2$ the proposed value $0$ and from process $p_4$ it receives $1$. Therefore $m=2$ and $v^{\star} = 0$ and if \textit{coin} happens to be $0$, then $p_3$ will decide on the value $0$. \\
			The process $p_4$ has received the proposed value $0$ from $p_2$ and $1$ from $p_1$ and $p_3$. Therefore $m=2$ and $v^{\star}=1$ and if \textit{coin} happens to be $1$, then $p_4$ will decide on $1$. \\
			In this scenario the processes $p_3$ and $p_4$ decide on different values which is violating the \textit{Agreement Property} and therefore this algorithm does not achieve the fault tolerance of $N > 3f$.
		\end{adjustwidth}
		\subsection*{(b) Guru Resilience?}
		\begin{adjustwidth}{2em}{2em}
			We assume that we have five processes $p_1$ to $p_5$, of which $p_1$ is the only faulty one. Furthermore $O=3$ and $G=2$. Every process needs to receive 4 \textsc{Vote} messages, in order to make a decision. Processes $p_2$ and $p_3$ proposes the value $0$ but processes $p_4$ and $p_5$ are proposing $1$. As before the message of a process to itself is delayed and does not affect the all in all decision. 
			In the first round process $p_2$ receives one message from $p_3$ with the proposed value of $0$ and the proposed value of $1$ from the other three processes. If \textit{coin} is $1$ the process $p_2$ will therefore decide on the value $1$. \\
			On the other hand process $p_4$ will receive $0$ from $p_1$ to $p_3$ and $1$ from $p_5$. If the \textit{coin} is equal to $0$ the process will decide on $0$. \\
			As in the previous exercise this violates the \textit{Agreement Property} and the algorithm does not achieve the fault tolerance of $N > 4f$.
		\end{adjustwidth}
		\subsection*{(c) Actual Resilience of \textit{Gurucoin}}
		\begin{adjustwidth}{2em}{2em}
			In order for a scenario to be able to violate the \textit{Agreement Property}, the following must happen:
			\begin{adjustwidth}{2em}{}
				One process which decides on a value, receives the proposed value of $0$ from $\lceil  \frac{N - f}{2} \rceil$ processes. The other $\lfloor \frac{N - f}{2} \rfloor$ non-faulty processes propose 1. In the end it has received \textsc{[Vote, r, 0]} from $\lceil  \frac{N - f}{2} \rceil + f$ processes and \textsc{[Vote, r, 1]} from $\lfloor  \frac{N - f}{2} \rfloor$ processes to totally receive $N-f$ messages. For another processes the faulty processes will \textsc{Vote} for the other value and the opposite would be the case.
			\end{adjustwidth}
			The two processes can only decide if $m \geq G$, therefore if
			\[
				\lceil  \frac{N - f}{2} \rceil + f < N - 3f
			\]
			holds it is not possible that the processes decide on different values. For $N = kf + 1$ we have:
			\begin{align*}
				\lceil  \frac{(k - 1)f + 1}{2} \rceil + f < (k - 3)f + 1 & , & k > 4 \texttt{ (from b)}
			\end{align*}
			We now can compute this for every $k$ until the inequality holds:
			\begin{enumerate}[]
				\item \hspace{0.1cm} \vdots 
				\item \underline{$k = 7$:}
				\[
					\lceil  \frac{6f + 1}{2} \rceil + f = 4f + 1 < 4f + 1 \textit{ is clearly wrong}
				\]
				\item \underline{$k = 8$:}
				\[
					\lceil  \frac{7f + 1}{2} \rceil + f < 5f + 1 
				\]
				\item \underline{$k = 9$:}
				\[
					\lceil  \frac{8f + 1}{2} \rceil + f = 5f + 1 < 6f + 1
				\]
			\end{enumerate}
			For every $k < 7$ the same would be the result as for $k = 7$ or $ k = 8$. With these computations we can see that $k$ must be $geq 9$ s.t. the inequality holds.
			\begin{enumerate}[]
				\item \textsc{Probabilistic Termination:} \\
				Clearly the algorithm will eventually terminate with a probability of 1.
				\item \textsc{Strong Validity:} \\
				If all processes propoose the same value, then $m \geq G$, thus all correct processes will decide on this value.
				\item \textsc{Integrity:} \\
				Since the value of \textit{decided} is set to \textsc{true} once a decision is made, a correct process cannot decide more than once.
			\end{enumerate}
		\end{adjustwidth}
	\end{adjustwidth}
\end{document}