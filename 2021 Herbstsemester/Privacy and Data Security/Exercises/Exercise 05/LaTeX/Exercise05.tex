\documentclass{report}

\usepackage{../../../../../LaTeX/marzstyle}

\runningheads{Privacy and Data Security}{Exercise 05}

\usepackage{csvsimple}
\UseRawInputEncoding
\usepackage{booktabs}

\setcounter{chapter}{5}

\begin{document}
	\section{$k$-Anonymity and and $l$-Diversity}
	\startsection
		\subsection{Table of 20 Random Selected Entries}
		\begin{adjustwidth}{-8em}{}
			\begin{tabular}{|l|l|l|l|l|l|l|l|}\hline
				\bfseries Name & \bfseries Surname & \bfseries E-Mail & \bfseries PLZ & \bfseries City & \bfseries System & \bfseries Semester & \bfseries Points
				\csvreader[head to column names]{../ex05_fake_dataset_comma.csv}{}{\\ \hline \csvcoli&\csvcolii&\small\csvcoliii&\csvcoliv&\csvcolv&\csvcolvi&\csvcolvii&\csvcolviii} % 
\\ \hline
			\end{tabular}
		\closesection
		
		\setcounter{subsection}{0}
		\renewcommand{\thesubsection}{\thesection.\alph{subsection}}
		
		\subsection{Define identifiers, two quasi-identifying attributes, and one sensitive attribute.}
		\startsubsection
			\textbf{Identifiers:} Name, Surname and E-Mail adress. \\
			\textbf{Quasi-Identifying Attributes:} PLZ and/or City (are dependent on each other), Points \\
			\textbf{Sensitive Attribute:} System
		\closesection
		\subsection{Sanitize this data by computing a 3-anonymized version of this data by hand.}
		\startsubsection
			I will define PLZ and Points-Ranges, in order to create clusters that share characteristics with each other, ensuring 3-Anonymity (see Table \ref{tab:3anonymDS}).
		\closesection
		\begin{table}\begin{center}
			\begin{tabular}{ccccc}
				\hline
				\textbf{PLZ-Range} && \textbf{Points-Range} && \textbf{System} \\
				\hline
				2500-3200 && 0-100 && MacOS \\
				&&&& Android \\
				&&&& Android \\
				\hline
				3200-5000 && 0-15 && Windows \\
				&&&& MacOS \\
				&&&& Linux \\
				\cmidrule{2-5}
				&& 16-30 && Linux \\
				&&&& Android \\
				&&&& MacOS \\
				\cmidrule{2-5}
				&& 31-50 && Android \\
				&&&& Windows \\
				&&&& MacOS \\
				&&&& Windows \\
				&&&& MacOS \\
				\cmidrule{2-5}
				&& 50-100 && iOS \\
				&&&& MacOS \\
				&&&& Windows \\
				&&&& Windows \\
				&&&& Android \\
				&&&& Android \\
				\hline
			\end{tabular}
		\caption{\label{tab:3anonymDS}3-Anonymity Dataset}
		\end{center}\end{table}
		\subsection{Does it become easier or more difficult to derive a 5-anonymized data set?}
		\startsubsection
			It is easy to determine a 5-anonymized data set, because one can take the subsets from 5.1.b and merge them together such that the set containing the least entries does contain at least 5 items.
		\closesection
		\subsection{Sanitize the entries to satisfy 3-diversity.}
		\startsubsection
			Using the previously created 3-Anonymity data set, by adjusting the ranges clusters can be created which contain at least 3 different System types, in order to ensure 3-Diversity (see Table \ref{tab:3diverseDS}).
		\closesection
		\begin{table}\begin{center}
			\begin{tabular}{ccccc}
				\hline
				\textbf{PLZ-Range} && \textbf{Points-Range} && \textbf{System} \\
				\hline
				2500-3750 && 0-15 && Windows \\
				&&&& MacOS \\
				&&&& MacOS \\
				&&&& Android \\
				\cmidrule{2-5}
				&& 16-30 && Linux \\
				&&&& Android \\
				&&&& MacOS \\
				\hline
				3150-3800 && 46-80 && MacOS \\
				&&&& Android \\
				&&&& iOS \\
				&&&& Android \\
				\cmidrule{2-5}
				&& 81-100 && MacOS \\
				&&&& Windows \\
				&&&& Windows \\
				&&&& Android \\
				&&&& Android \\
				\hline
				3450-5000 && 31-45 && Android \\
				&&&& Windows \\
				&&&& MacOS \\
				&&&& Windows \\
				\hline
			\end{tabular}
		\caption{\label{tab:3diverseDS}3-Diversity Dataset}
		\end{center}\end{table}
	\closesection
\end{document}