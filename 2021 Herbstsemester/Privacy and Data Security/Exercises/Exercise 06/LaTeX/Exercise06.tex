\documentclass{report}

\usepackage{../../../../../LaTeX/marzstyle}

\runningheads{Privacy and Data Security}{Exercise 06}

\setcounter{chapter}{6}

\begin{document}
	\section{$k$-Anonymity and and $l$-Diversity}
	\startsection
		\subsection{Distinct $l$-Diversity}
		\startsubsection
			For the distinct $l$-Diversity it is trivial to show that $l$-divers sanitized dataset also satisfies $l$-Anonymity. $l$-Diversity says that every subset contains at least $l$ different values of the sensitive attributes. Because this is a given each subset must at least contain $l$ different entries in order to satisfy this statement. This is of course the definition of $l$-Anonymity which is therefore also satisfied.
		\closesection
		\subsection{Probabilistic $l$-Diversity}
		\startsubsection
			The definition of probabilistic $l$-Diversity is that any value has a relative frequency of at most $\frac{1}{l}$. Therefore if a sanitized dataset fulfills this requirement each subset will have at least $l$ entries, which is also satisfying the $l$-Anonymity criteria.
		\closesection
	\closesection
	
	\section{Implementing $k$-Anonymity with the Mondrian Algorithm}
	\startsection
		\renewcommand{\thesubsection}{\thesection.\alph{subsection}}
		\hfill \vspace{-1cm}
		
		\subsection{Normalized Certainty Penalty}
		\textit{Using only PLZ and points as QI (and represented as numerical attributes), and with system as S, compute at least a 3-, 5-, and 10-anonymization of the dataset and report its NCP. \\ What is the NCP of a 1-anonymization and that of a 74-anonymization?}
		\startsubsection
			For the different k-anonymizations we get the NCP values:
			\begin{enumerate}[(1)]
				\item 3-Anonymity: 16,40\%
				\item 5-Anonymity: 26,49\%
				\item 10-Anonymity: 48,08\%
			\end{enumerate}
			The NCP of a 1-anonymization would be 0,00 \% and for 74-anonymization the NCP would be 100,00\%.
		\closesection
		
		\subsection{Normalized Certainty Penalty of Permutations}
		\textit{Permute the dataset randomly (e.g., calling shuf) and observe the outcome. Extend the algorithm (using randomization) to compute improved 3-, 5-, and 10-anonymizations, that is, achieving better NCP than under a).}
		\startsubsection
			When shuffling the data beforehand the NPCs change in a range of $\pm \textit{a couple of} \%$. \\
			In order to automate the whole process a new class was created with the following code:
			\begin{minted}{python}
import os

if __name__ == '__main__':
    for i in range(10):
        os.system("python anonymizer.py a d 5 | findstr NCP") # for windows
        os.system("python anonymizer.py a d 5 | grep NCP") # for linux
			\end{minted}
			With this command we can run the anonymizer algorithm several times and get the lowest NCP. The shuffling of the data is done within the read\_dataset script:
			\begin{minted}{python}
def read_data():
	data = []
	data_file = open('data/ex06_fake_dataset.csv', 'rU')
	#######################
	# parseFile into data #
	#######################
	random.shuffle(data)
	return data
			\end{minted}
			This is only possible if we use "Ort" as a QI instead of the PLZ.
		\closesection
	\closesection
\end{document}