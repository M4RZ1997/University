\documentclass{report}

\usepackage{../../../../../LaTeX/marzstyle}

\runningheads{Privacy and Data Security}{Exercise 06}

\setcounter{chapter}{6}

\begin{document}
	\section{$k$-Anonymity and and $l$-Diversity}
	\startsection
	\closesection
	
	\section{Implementing $k$-Anonymity with the Mondrian Algorithm}
	\startsection
		\renewcommand{\thesubsection}{\thesection.\alph{subsection}}
		\hfill \vspace{-1cm}
		
		\subsection{Normalized Certainty Penalty}
		\textit{Using only PLZ and points as QI (and represented as numerical attributes), and with system as S, compute at least a 3-, 5-, and 10-anonymization of the dataset and report its NCP. \\ What is the NCP of a 1-anonymization and that of a 74-anonymization?}
		\startsubsection
		\closesection
		
		\subsection{Normalized Certainty Penalty of Permutations}
		\textit{Permute the dataset randomly (e.g., calling shuf) and observe the outcome. Extend the algorithm (using randomization) to compute improved 3-, 5-, and 10-anonymizations, that is, achieving better NCP than under a).}
		\startsubsection
		\closesection
	\closesection
\end{document}