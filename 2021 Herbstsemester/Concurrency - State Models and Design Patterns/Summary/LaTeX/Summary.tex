\documentclass{report}

\usepackage{../../../../LaTeX/marzstyle}

\author{\\ Marcel Matti Zauder}
\runningheads{Concurrency - State Models and Design Patterns}{Summary}

\fancypagestyle{plain}{} % fancyhdr also on first Chapter pages

\titleformat{\chapter}[display]   
{\normalfont\huge\bfseries}{\chaptertitlename\ \thechapter}{20pt}{\Huge}   
\titlespacing*{\chapter}{0pt}{-20pt}{30pt}


\begin{document}
	\chapter{Introduction}	
	\section{Concurrency and Parallelism}
	\startsection
		A concurrent program, unlike a sequential program, has multiple threads of control which may be executed as parallel processes. \\
		A \textit{thread} and \textit{process} in Computer Science have multiple, overlapping, and inconsitent meanings. Generally, a \textit{"process"} is associated with an instance of a running software program, which contains its own address space, whereas a \textit{"thread"} refers to a \textit{"thread of control"} within a given process and therefore does not contain its own address space but shares the space of the process with multiple other threads. \\ \\
		In order to execute a concurrent program different approaches can be chosen:
		\begin{enumerate}[-]
			\item \textsc{Multiprogramming}: \\
			The processes are sharing one or more processors.
			\item \textsc{Multiprocessing}: \\
			Each process runs on its own processor with access to a shared memory.
			\item \textsc{Distributed Processing}: \\
			Each process runs on its own processor while being connected to others via a network.
		\end{enumerate}
	\closesection
	\section{\textsc{Challenges}: Safety and Liveness}
	\startsection
	\closesection
	\section{Expressing Concurrency}
	\startsection
		\subsection{Process Creation}
		\startsubsection
		\closesection
		\subsection{Communication and Synchronization}
		\startsubsection
		\closesection
	\closesection
\end{document}