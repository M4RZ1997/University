\documentclass{report}

\usepackage{../../../../../LaTeX/marzstyle}

\author{Marcel \textsc{Zauder} 16-124-836 \\
	Pascal \textsc{Gerig} 16-104-721}

\runningheads{Concurrency}{Exercise 09}

\setcounter{chapter}{9}

\begin{document}
	\section{Several Questions}
	\startsection
		\begin{enumerate}[a)]
			\item \textit{What criteria might you use to prioritize threads (list at least 5 different criteria)?}
			\begin{enumerate}[(i)]
				\item Based on the due time/desired completion time of a process.
				\item Based on any representation of priorization, i.e. task priority, cost etc.
				\item Based on the time the service was requested (FIFO).
				\item Based on the expected time a thread needs to be served.
				\item Based on how many other tasks are waiting for the result of  this particular task.
			\end{enumerate}
			\item \textit{What are different possible definitions of fairness (list at least 3 different definitions)?}
			\begin{enumerate}[(i)]
				\item \textit{Weak Fairness} \\
				A transition or a process should not wait an unbounded amount of time to execute if it is enabled continuously.
				\item \textit{Strong Fairness} \\
				A transition or a process should not wait an unbounded amount of time to execute if it is enabled infinitely often.
				\item \textit{Linear Waiting} \\
				A transition or a process making a request will be served before any other process was served more than once.
				\item \textit{FIFO} \\
				A transition or a process making a request will be served before any other process that made a request after that.
			\end{enumerate}
			\item \textit{What are Pass-Throughs?} \\
			With \textit{Pass-Throughs} the host maintains a set of immutable references to helper objects. All messages are then relayed to these within unsynchronized methods.
			\item \textit{What is Lock-Splitting?} \\
			In \textit{Lock-Splitting} instead of using the same lock for each method of a class an individual lock is created for each method, i.e. one lock for writing and one lock for reading.
			\item \textit{When should you consider using optimistic methods (list at least 3 different enablers)?}
			\begin{enumerate}[(i)]
				\item Clients can tolerate either failure or retries.
				\item Avoidance or Coping of livelocks.
				\item Before a failure occurs the program can rollback into a non-fail state and try again.
				\item The chance of a collition is negligible.
			\end{enumerate}
		\end{enumerate}
	\closesection
	
	\newpage
	\setcounter{section}{2}	
	\section{Additional Several Questions}
	\startsection
		\begin{enumerate}[a)]
			\item \textit{How do threads waiting in a Thread.join() loop get aware of that thread’s termination?} \\
			When \textit{t.join()} is called the thread/process which is calling that method is waiting while the referenced thread is \textsc{alive}. After \textit{t} terminates the \textit{join()} will also leave the wait-loop and terminate. It is also possible to wait for a given amount of time and if the referenced thread throws an \textit{InterruptedException} the \textit{join}-method will as well.
			\item \textit{How could you optimize the code below?}
			\begin{minted}{java}
Thread t = new Thread(new Runnable() {
	@Override
	public void run() {
		<insert your code here>
	}
});
t.start();
t.join();
			\end{minted}
			Because the thread is started and directly afterwards it is joined, there is no need for a thread implementation, but the code in the run method itself can just be executed directly.
			\item \textit{Are String objects in Java mutable or immutable? Justify your answer!} \\
			A \textsc{String} object in Java is immutable, because once it is generated this particular instance cannot be changed. In order to create strings on runtime, A StringBuilder should be used. Also there does not exist any method call on a \textsc{String} object that changes the object itself but creates a new modified \textsc{String} object.
			\item \textit{Does the FSP progress property below enforce fairness? Justify your answer!}
			\begin{minted}{shell}
progress HeadsOrTales = {head, tale}
			\end{minted}
			No, it does not, because it only ensures that either the action \textit{head} or \textit{tale} can be executed at any time, but it does not ensure that it is actually executed.
		\end{enumerate}
	\closesection
\end{document}