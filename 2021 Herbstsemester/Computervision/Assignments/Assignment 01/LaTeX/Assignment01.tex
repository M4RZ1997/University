\documentclass{report}

\usepackage{../../../../../LaTeX/marzstyle}

\runningheads{Computervision}{Assignment 01}

\setcounter{chapter}{1}


\begin{document}
	\section{Blur Kernel}
	\startsection
		Last digit of matriculation number: 6 $\Rightarrow$ $6 \ mod \ 4 = 2$ \\
		$\Rightarrow \ k_2 = \begin{pmatrix} \frac{1}{2} & 0 \\ 0 & \frac{1}{2} \end{pmatrix}$
	\closesection

	\section{Discretization of E}
	\startsection
		\textit{Finite difference approximation of the objective function E: Choose forward differences for the discretization.
In particular, use eq. (6) for the Gaussian prior implementation and derive the corresponding discretization
for the anisotropic prior. Write the main steps of your calculations in the report.}
		\subsection{Discretization of anisotropic total variation regularization term}
		\startsubsection
			The second term is defined as:
			\[ R[u] \ = \ | \nabla u | _1 \ = \ \sum_{i=0}^{m-1} \sum_{j=0}^{n-1} | \nabla u[i,j] | _1 \hspace{2cm} (5)\]
			The analytic derivative with forward differences is defined as follows:
			\[ \nabla u[x] \ = \ \lim_{\epsilon \rightarrow 0} \frac{u[x + \epsilon] - u[x]}{\epsilon} \]
			The grid step $\epsilon$ is set to 1, because it is the smalles possible grid step:
			\[ \nabla u[i,j] \ \simeq \ \begin{bmatrix} u[i+1,j] - u[i,j] \\ u[i,j+1] - u[i,j] \end{bmatrix} \]
			Because we are applying the L1-norm on this derivative we can write:
			\[ | \nabla u[i,j] | _1 \ \simeq \ | u[i+1,j] - u[i,j] | + | u[i,j+1] - u[i,j] | \]
			Now we need to make sure that the formula doesn't reach "out-of-bounds":
			\subsubsection{1. Every pixel except the ones with $u[i,j] = u[m-1,j]$ and $u[i,j] = u[j,n-1]$}
			\[
				\Rightarrow \sum_{i=0}^{m-2} \sum_{j=0}^{n-2} | u[i+1,j] - u[i,j] | + | u[i,j+1] - u[i,j] |
			\]
			\subsubsection{2. Every pixel with $u[i,j] = u[m-1,j]$}
			\[
				\Rightarrow \sum_{j=0}^{n-2} | u[m-1,j+1] - u[m-1,j] |
			\]
			\subsubsection{3. Every pixel with $u[i,j] = u[j,n-1]$}
			\[
				\Rightarrow \sum_{i=0}^{m-2} | u[i+1,n-1] - u[i,n-1] |
			\]
			\subsubsection{Joined together this will give us the discretized regularization term}
				\[
					\hspace{-2.5cm} R[u] \ = \ | \nabla u | _1 \ = \ \sum_{i=0}^{m-2} \sum_{j=0}^{n-2} | u[i+1,j] - u[i,j] | + | u[i,j+1] - u[i,j] | + \sum_{j=0}^{n-2} | u[m-1,j+1] - u[m-1,j] | + \sum_{i=0}^{m-2} | u[i+1,n-1] - u[i,n-1] | \hspace{1.5cm} (7)
				\]
		\closesection
		\subsection{Simplification of term (3) $| u \ast k_2 -g |$}
		\startsubsection
			\[
				| u \ast k_2 - g | ^2 \ = \ \sum_{i=0}^{m-1} \sum_{j=0}^{n-1} | g[i,j] - \sum_{p=0}^{1} \sum_{q=0}^{1} k(p,q) \ u[i-p+1, j-q+1] | _2 ^2 \hspace{2cm} (3)
			\]
		\closesection
	\closesection
	
	\section{Gradient Calculations}
	\startsection
		\begin{enumerate}[(a)]
			\item \textit{Compute the gradient $\nabla_u E$ at pixels inside the image, i.e. $1 \leq i \leq m-2$ and $1 \leq j \leq n-2$} \\
			\item \textit{Compute the gradient $\nabla_u E$ at the four corners, i.e. $(i,j) \in \{ (0,0); (0,n-1);(m-1,0);(m-1,n-1)) \}$} \\
			\item \textit{Compute the gradient $\nabla_u E$ at pixels on the upper edge, i.e. $i = 0$} \\
			\item \textit{Compute the gradient $\nabla_u E$ at pixels on the left edge, i.e. $j = 0$} \\
			\item \textit{Compute the gradient $\nabla_u E$ at pixels on the right edge, i.e. $j = n - 1$} \\
			\item \textit{Compute the gradient $\nabla_u E$ at pixels on the lower edge, i.e. $i = m-1$}
		\end{enumerate}
	\closesection
\end{document}