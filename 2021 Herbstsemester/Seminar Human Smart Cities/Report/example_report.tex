\documentclass[a4paper,12pt]{report}
\usepackage{unifrsr}
\usepackage{hsc_texstyle}

\begin{document}
	\seminartitle{Human Smart Cities Seminar} % The title of the seminar

	\title{Wisdom of the Crowds} % The title of your project

	\author{Pascal Gerig (16-104-721)\thanks{\email{pascal.gerig@students.unibe.ch}, University of Bern}
   		\and Lorenzo Wipfli  (13-933-262)\thanks{\email{lorenzo.wipfli@students.unibe.ch}, University of Bern}
   		\and Marcel Zauder  (16-124-836)\thanks{\email{marcel.zauder@students.unibe.ch}, University of Bern}
   	}	% Note: XX-XXX-XXX is the student's matriculation number
   	% The author(s), separated by \and

	\supervisor{Prof. Edy Portmann} % Name of the supervisor

	\assistant{Moreno Colombo \and Jhonny Pincay} %Name of the assistant(s)

	\date{\today} % Note: if this is left out, today's date will be used, this is the submission date!

	\maketitle

	\begin{abstract}
		This package, \textsf{unifrsr}, allows the easy creation of seminar
		reports using standard \LaTeX. It should be the last
		package loaded, to ensure that nothing overrides the page layout.

		Note that utility commands are available: 
		\begin{itemize}
			\item \verb+\seminartitle+ which specifies, optionally, the title of the seminar the report is being written for
			\item \verb+\supervisor+ which specifies the name of the professor supervisor of the project
			\item \verb+\assistant+ which specifies, optionally, the name of the assistants of the seminar, separated by the command \verb+\and+
		\end{itemize}

		\keywords{Seminar report, Human-IST Research Institute}
	\end{abstract}

	\tableofcontents

	\chapter{Introduction}
		This is an example of a section. As you see, it is just a standard {\LaTeX} section. Here is a footnote%
		\footnote{This is a footnote}.

		\section{Subsection of introduction}
		\startsection
			It's all standard \LaTeX, as you can see. This is a very nice paper:~\cite{zadeh}.
		\closesection



	\newpage
	\addcontentsline{toc}{chapter}{Bibliography}
	\bibliography{references}
		\bibliographystyle{plain}
\end{document}
