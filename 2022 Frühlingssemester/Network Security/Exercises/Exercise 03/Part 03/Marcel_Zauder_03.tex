\documentclass{report}

\usepackage{../../../../../LaTeX/marzstyle}

\newcommand{\exercisenr}{3}

\runningheads{Network Security}{Exercise 0\exercisenr}

\setcounter{chapter}{\exercisenr}
\setcounter{section}{2}


\begin{document}
	\section{Question 3}
	\startsection
		\renewcommand{\thesubsection}{\thesection.\Alph{subsection}}
		\subsection{In the Diffie-Hellman protocol, each participant selects a secret number $x$ and sends the other participant $g^x \ mod \ p$ for some public number $g$. What would happen if the participants sent each other $x^g$ for some public number $g$ instead? Give at least one method Alice and Bob could use to agree on a key. Can Eve break your system without finding the secret numbers? Can Eve find the secret number}
		\startsubsection
			As $g$ is a publicly known generator, Eve can easily compute the secret number $x$ as the "Indiscrete Logarithm Problem" is not hard - therefore the security is not given anymore. \\
			For example Alice and Bob can agree on the public number $g = 4$. Then each of them chooses a secret number - i.e. Alice chooses 5 and Bob 7 - and then computes $x^g$ - which is $5^4 \ = \ 625$ and $7^4 \ = \ 2401$, respectively. Both Alice and Bob exchange these numbers via a public medium which Eve can eavesdrop. As previously mentioned the "Indiscrete Logarithm Problem" is not hard and with the knowledge of the public number Eve can compute both secrets.
		\closesection
	\closesection
\end{document}