\documentclass{report}

\usepackage{../../../../../LaTeX/marzstyle}

\newcommand{\exercisenr}{3}

\runningheads{Network Security}{Exercise 0\exercisenr}

\setcounter{chapter}{\exercisenr}
\setcounter{section}{2}


\begin{document}
	\section{Question 3}
	\startsection
		\renewcommand{\thesubsection}{\thesection.\Alph{subsection}}
		\subsection{In the Diffie-Hellman protocol, each participant selects a secret number $x$ and sends the other participant $g^x \ mod \ p$ for some public number $g$.}
		\startsubsection
			\subsubsection{What would happen if the participants sent each other $x^g \ mod \ p$ instead}
			\startsubsection
				As $g$ is a publicly known generator, Eve can easily compute the secret number $x$ as the "Indiscrete Logarithm Problem" is not hard - therefore the security is not given anymore.
			\closesection
			\subsubsection{Suggest a method that the participants could apply for generating a common key (using the $x^g \ mod \ p$ approach)}
			\startsubsection
				Both Bob and Alice exchange $x^g \ mod \ p$ and $y^g \ mod \ p$ and then both can compute $x^g * y^g \ mod \ p$, which will be the same as $(xy)^g \ mod \ p$.
			\closesection
			\subsubsection{Can Eve break your system without finding the secret numbers?}
			\startsubsection
				Yes, Eve can use the eavesdropped information she got and multiply both of those modulo $p$ and gets the same solution as Alice and Bob are receiving by following the mentioned protocol.
			\closesection
			\subsubsection{Can Eve find the secret number?}
			\startsubsection
				Yes, as finding the secret numbers of Alice and Bob is known as the "Indiscrete Logarithm Problem" which is not hard.
			\closesection
		\closesection
	\closesection
\end{document}