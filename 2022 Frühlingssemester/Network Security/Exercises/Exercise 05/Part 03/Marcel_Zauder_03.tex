\documentclass{report}

\usepackage{../../../../../LaTeX/marzstyle}

\newcommand{\exercisenr}{5}

\runningheads{Network Security}{Exercise 0\exercisenr}

\setcounter{chapter}{\exercisenr}
\setcounter{section}{2}


\begin{document}
	\section{Question 3}
	\startsection
		\renewcommand{\thesubsection}{\thesection.\Alph{subsection}}
		\subsection{Is scenario 2 safe against man-in-the-middleattacks? Why?}
		\startsubsection
			No, it is not, as a malicious entity/adversary $E$ can intercept the messages sent between $A$ and $B$. A following scenario can be used as an example for such an attack:
			\begin{enumerate}
				\item $A$ generates public/private key pair $\{ PU_a , PR_a \}$ and sends $PU_a \| ID_A$ to $B$
				\item $E$ intercepts message, creates its own public/private key pair $\{ PU_c , PR_c \}$ and sends $PU_c \| ID_A$ to $B$
				\item $B$ thinks the message comes from $A$ and performs the encryption using a generated secret key $K_s$ and sends $E(PU_c, K_s)$ to $A$
				\item $E$ intercepts message decrypts message from $B$, learning $K_s$
				\item $E$ sends $E(PU_a, K_s)$ to $A$
			\end{enumerate}
			As $E$ knows $K_s$ the adversary can eavesdrop the messages sent between $A$ and $B$ and is able to decrypt the messages using $K_s$.
		\closesection
		\subsection{In scenario 1, is each side confident about the authenticity of the other side? Why?}
		\startsubsection
			As in step 2 the nonce $N1$ is being send concatenated with nonce $N2$ it ensures $A$ that this message comes from $B$ as only B is able to decrypt $N1$ from the first message. As in step3 A uses the public key of B to encrypt $N2$, $B$ can be assure that it is communicating with $A$. Then $A$ sends the message $E(PU_b, E(PR_a, K_s))$ to $B$ which is ensuring that only $B$ can read it as it is encrypted using $B$'s public key and that this message comes from A as $K_s$ is encrypted using $A$'s private key. Hence, this scenario ensure confidentiality and authenticity.
		\closesection
		\subsection{In scenario 1, assume that in step 2 only nonce 2 is being transmitted (and not nonce 1). In the end of step 4, which side is ensured about the identity of the other side? Why?}
		\startsubsection
			As written before $N1$ is being sent to $A$ to ensure that the Responder $B$ is actually the "wanted" responder. Therefore, in the end only $B$ can be assure that it was talking with the "wanted" Initiator $A$.
		\closesection
	\closesection
\end{document}