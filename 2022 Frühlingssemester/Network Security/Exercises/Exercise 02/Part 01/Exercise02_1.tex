\documentclass{report}

\usepackage{../../../../../LaTeX/marzstyle}

\runningheads{Network Security}{Exercise 02}

\setcounter{chapter}{2}


\begin{document}
	\section{Question 1}
	\startsection
		\renewcommand{\thesubsection}{\thesection.\Alph{subsection}}
		\subsection{Provide a quick explanation why the following statements are True or False:}
		\startsubsection
			\subsubsection{Symmetric encryption is a crypto-mechanism where encryption and decryption are performed using different keys.}
			\startsubsection
				\textbf{False}, in symmetric encryption only one key is used for both encrypting and decrypting a message.
			\closesection
			\subsubsection{With the use of symmetric encryption, the principal security problem is maintaining the secrecy of the key.}
			\startsubsection
				\textbf{True}, in symmetric encryption only one key is used and therefore it must be only available only to the communcating parties as every procedure for encryption/decryption can be performed with this single key.
			\closesection
			\subsubsection{The process of converting from plaintext to ciphertext is known as deciphering or decryption.}
			\startsubsection
				\textbf{False}, it is called enciphering or encryption. The process of converting from ciphertext to plaintext is known as deciphering or decryption.
			\closesection
			\subsubsection{The algorithm will produce a different output depending on the specific secret key being used at the time. The exact substitutions and transformations performed by the algorithm depend on the key.}
			\startsubsection
			\closesection
			\subsubsection{When using symmetric encryption it is very important to keep the algorithm secret.}
			\startsubsection
				\textbf{False},
			\closesection
			\subsubsection{Ciphertext generated using a computationally secure encryption scheme is impossible for an opponent to decrypt simply because the required information is not there.}
			\startsubsection
			\closesection
		\closesection
	\closesection
\end{document}