\documentclass{report}

\usepackage{../../../../../LaTeX/marzstyle}

\newcommand{\exercisenr}{11}

\runningheads{Network Security}{Exercise \exercisenr}

\setcounter{chapter}{\exercisenr}
\setcounter{section}{5}


\begin{document}
	\section{Question 6}
	\startsection
		\renewcommand{\thesubsection}{\thesection.\Alph{subsection}}
		\subsection{Which security services are provided by DNSSEC? By which security mechanism(s) are they provided? (X.800)}
		\startsubsection
			DNSSEC adds two security services:
			\begin{enumerate}
				\item \textit{Data Origin Authentication} \\
				Ensures the receiver/resolver to verify that the data originated from the zone he believes it comes from.
				\item \textit{Data Integrity Protection} \\
				Ensures the receiver/resolver that the data was not modified during transfer.
			\end{enumerate}
		\closesection
		\subsection{Is DNSSEC enabled for unibe.ch? Describe how you found out.}
		\startsubsection
			Using the "\textit{dig ds unibe.ch +short}" command I could determine that DNSSEC is enabled for unibe.ch. Also various internet websites concluded in the same result.
		\closesection
	\closesection
\end{document}