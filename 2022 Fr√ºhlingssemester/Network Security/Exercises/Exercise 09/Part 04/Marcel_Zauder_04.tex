\documentclass{report}

\usepackage{../../../../../LaTeX/marzstyle}

\newcommand{\exercisenr}{9}

\runningheads{Network Security}{Exercise 0\exercisenr}

\setcounter{chapter}{\exercisenr}
\setcounter{section}{3}


\begin{document}
	\section{Question 4}
	\startsection
		\renewcommand{\thesubsection}{\thesection.\Alph{subsection}}
		\subsection{Explain which cellular network generation is most suited for IoT applications. Why?}
		\startsubsection
			For IoT applications the currently best cellular network generation would be 5G, as older generations like 2G and 3G are not able to provide every demand necessary (Source 5G Explained pp. 205 ff. - also for the rest of this exercise). Also 5G implements different standards which were developed with the concept of IoT in mind:
			\begin{enumerate}[-]
				\item updated standards in network security
				\item Virtualization, Edge Computing, and Network Slicing enabled
				\item data transfer with large number of devices involved (with the implementation of mIoT building blocks)
				\item different ways of accessing public networks (direct, indirect, with a group)
			\end{enumerate}
		\closesection
	\closesection
\end{document}