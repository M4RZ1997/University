\documentclass{report}

\usepackage{../../../../../LaTeX/marzstyle}

\newcommand{\exercisenr}{11}

\runningheads{Network Security}{Exercise \exercisenr}

\setcounter{chapter}{\exercisenr}
\setcounter{section}{4}


\begin{document}
	\section{Question 5}
	\startsection
		\renewcommand{\thesubsection}{\thesection.\Alph{subsection}}
		\subsection{S/MIME provides authentication, confidentiality, compression, and email compatibility. Which ones of the following statements about these four services are true and which ones are false? Justify.}
		\startsubsection
			\subsubsection{S/MIME uses public key cryptography for content encryption in order to ensure confidentiality.}
			\startsubsection
				\textbf{False}, the key used is a symmetric key, which is only used once. For each message this key is generated from anew and encrypted with the use of the receiver's public key.
			\closesection
			\subsubsection{S/MIME requires that the signing is done first followed by the message encryption.}
			\startsubsection
				\textbf{True}, as can be seen on slide 25 of the lecture.
			\closesection
			\subsubsection{S/MIME uses X.509 public-key certificates.}
			\startsubsection
				\textbf{True}, as on slide 31 of the lecture it can be seen that S/MIME uses public-key certificates that conform to version 3 of X.509.
			\closesection
			\subsubsection{Lossy compression can be applied in any order with respect to the signing and message encryption operations.}
			\startsubsection
				\textbf{False}, as lossy compression leads to the need to perform the compression first and then the signing.
			\closesection
			\subsubsection{S/MIME provides 7-bit encoding for converting a stream of 8-bit octets to a stream of ASCII characters while some electronic mail systems accept only blocks of ASCII characters.}
			\startsubsection
				\textbf{False}, S/MIME provides the service of converting a raw 8-bit binary stream to a stream of printable ASCII characters. This service is called 7-bit encoding.
			\closesection
		\closesection
	\closesection
\end{document}