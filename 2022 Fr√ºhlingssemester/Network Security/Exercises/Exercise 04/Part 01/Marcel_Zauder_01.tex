\documentclass{report}

\usepackage{../../../../../LaTeX/marzstyle}

\newcommand{\exercisenr}{4}

\runningheads{Network Security}{Exercise 0\exercisenr}

\setcounter{chapter}{\exercisenr}
\setcounter{section}{0}


\begin{document}
	\section{Question 1}
	\startsection
		\renewcommand{\thesubsection}{\thesection.\Alph{subsection}}
		\subsection{Provide a quick explanation for the statements that are \textsc{False}:}
		\startsubsection
			\subsubsection{The principle requirement of PRNG is that the generated number stream be unpredictable.}
			\startsubsection
				\textbf{True}
			\closesection
			\subsubsection{With true random sequences each number is statistically independent of others and therefore unpredictable.}
			\startsubsection
				\textbf{True}
			\closesection
			\subsubsection{The pseudorandom number generator may simply involve conversion of an analog source to a binary output.}
			\startsubsection
				\textbf{False}, this is only true for a \textbf{true} random number generator, as can be seen on slide 7 from the lecture.
			\closesection
			\subsubsection{Examples of a pseudorandom function are decryption keys and nonces.}
			\startsubsection
				\textbf{False}, 
			\closesection
			\subsubsection{A widely used technique for pseudorandom number generation is an algorithm known as the linear congruential method.}
			\startsubsection
				\textbf{True}
			\closesection
			\subsubsection{The security of Blum, Blum, Shubis not based on the difficulty of factoring n.}
			\startsubsection
				\textbf{False}, it is dependant on the fact the factoring n is difficult.
			\closesection
		\closesection
	\closesection
\end{document}