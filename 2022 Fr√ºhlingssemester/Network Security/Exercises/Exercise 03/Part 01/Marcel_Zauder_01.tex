\documentclass{report}

\usepackage{../../../../../LaTeX/marzstyle}

\newcommand{\exercisenr}{3}

\runningheads{Network Security}{Exercise 0\exercisenr}

\setcounter{chapter}{\exercisenr}
\setcounter{section}{0}


\begin{document}
	\section{Question 1}
	\startsection
		\renewcommand{\thesubsection}{\thesection.\Alph{subsection}}
		\subsection{Provide a quick explanation why the following statements are True or False:}
		\startsubsection
			\subsubsection{Asymmetric encryption can be used for confidentiality but not for authentication}
			\startsubsection
				\textbf{False}, if one uses the private key for encryption - or the private key to create a signature - authenticity can also be ensured.
			\closesection
			\subsubsection{In asymmetric encryption, plaintext is transformed into ciphertext using two keys and a decryption algorithm.}
			\startsubsection
				\textbf{False}, one key is used for encryption and a different but related key is used for decryption.
			\closesection
			\subsubsection{Much of the theory of public-key cryptosystems is based on number theory.}
			\startsubsection
				\textbf{True}, especially modulo operations are often used in creating the key, enciphering and deciphering.
			\closesection
			\subsubsection{A public-key encryption scheme is not vulnerable to a brute-force attack.}
			\startsubsection
				\textbf{False}, as the public key and the private key are mathematically related it might be easier to find this relation than brute-forcing all different keys.
			\closesection
			\subsubsection{The defense against the brute-force approach for RSA is to use a large key space.}
			\startsubsection
				\textbf{True}, as brute-forcing is a "viable" option the prime numbers must be very large - making also the key space very large - in order to have the same level of security as symmetric encryption.
			\closesection
		\closesection
	\closesection
\end{document}