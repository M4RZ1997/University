\documentclass{report}

\usepackage{../../../../../LaTeX/marzstyle}

\newcommand{\exercisenr}{3}

\runningheads{Network Security}{Exercise 0\exercisenr}

\setcounter{chapter}{\exercisenr}
\setcounter{section}{1}


\begin{document}
	\section{Question 2}
	\startsection
		\renewcommand{\thesubsection}{\thesection.\Alph{subsection}}
		Suppose Bob uses the RSA cryptosystem with a very large modulus $n$ for which the factorization cannot be found in a reasonable amount of time. Suppose Alice sends a message to Bob by representing each alphabetic character as an integer between 0 and 25 (A$\rightarrow$0,…,Z$\rightarrow$25) and then encrypting each number separately using RSA with large $e$ and large $n$. 
		\subsection{Is this method secure?}
		\startsubsection
			No, it is not secure!
		\closesection
		\subsection{If not, describe the most efficient attack against this encryption method.}
		\startsubsection
			The simplest attack would be that an intruder computes $m^e \ mod \ n$, for all possible values of $m$. This does not take much time because $m$ has only 26 values. Then the intruder can create a decryption table in which the decryption of $m^e \ mod \ n$ is $m$.
		\closesection
	\closesection
\end{document}