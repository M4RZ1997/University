\documentclass{report}

\usepackage{../../../../../LaTeX/marzstyle}

\runningheads{Network Security}{Exercise 02}

\setcounter{chapter}{2}
\setcounter{section}{4}


\begin{document}
	\section{Question 5}
	\startsection
		\renewcommand{\thesubsection}{\thesection.\Alph{subsection}}
		\subsection{Is it possible to perform encryption and decryption operations in parallel on multiple blocks of plain text in CBC mode? Justify your answer.}
		\startsubsection
			No, it is not because the result of the previous block is always carried over in order to encrypt/decrypt the next block.
		\closesection
		\subsection{If a bit error occurs in the transmission of a cipher text character in 8-bit CFB mode, how far does the error propagate?}
		\startsubsection
			The error will obviously affect the decryption of the associated ciphertext block. Furthermore, this error block will stay in the initialization vector for another $64/8 = 8$ blocks, hence, affecting in total 9 blocks.
		\closesection
		\subsection{CBC mode}
		\startsubsection
			\subsubsection{$C_1$ corrupts $P_1$ and $P_2$. Are any blocks beyond $P_2$ affected?}
			\startsubsection
				No, because the blocks beyond do not rely on the information of  $C_1$.
			\closesection
			\subsubsection{Given a bit error in the source version of $P_1$. Through how many ciphertext blocks is the error propagated?}
			\startsubsection
				This error will affect every ciphertext block that will be encrypted because the previous generated ciphertext is used to encrypt the next block and therefore the error is always propagated.
			\closesection
			\subsubsection{What is the effect at the receiver?}
			\startsubsection
				Because the error was always propagated, the following blocks will be decrypted to the original blocks because the falsely encrypted blocks are then decrypted in a false way as the error propagated through all blocks and therefore the correct blocks are deciphered.
			\closesection
		\closesection
	\closesection
\end{document}