\documentclass{report}

\usepackage{../../../../../LaTeX/marzstyle}

\newcommand{\exercisenr}{12}

\runningheads{Network Security}{Exercise \exercisenr}

\setcounter{chapter}{\exercisenr}
\setcounter{section}{0}


\begin{document}
	\section{Question 1}
	\startsection
		\renewcommand{\thesubsection}{\thesection.\Alph{subsection}}
		\subsection{Decide, which incoming packets will be accepted, and which will be dropped. Specify, which rule will accept/drop each packet.}
		\begin{enumerate}[-]
			\item src: 142.8.5.11, dst: 141.9.7.11:22, tcp \\ \textbf{dropped}, Rule 7
			\item src: 152.4.8.2, dst: 141.9.7.12, icmp \\ \textbf{accepted}, Rule 3
			\item src: 142.5.17.99, dst: 141.9.7.111:443, tcp \\ \textbf{dropped}, Rule 1
			\item src: 141.9.7.215, dst: 141.9.7.11:443, tcp \\ \textbf{accepted}, Rule 6
			\item src: 2003:4860:4860::8844, dst: 2016:837:62f::b8, dstport: 53, udp \\ As IPv6 packets are not handled at all by generic \textit{iptables} this packet is just \textbf{accepted}.
		\end{enumerate}
		\subsection{One of the packets from question 1 A is malicious}
		\startsubsection
			\subsubsection{Which one?}
			\startsubsection
			\closesection
			\subsubsection{What’s malicious about it?}
			\startsubsection
			\closesection
			\subsubsection{Add a rule that would block such malicious packets}
			\startsubsection
			\closesection
			\subsubsection{In which place will you add the new rule to the table?}
			\startsubsection
			\closesection
		\closesection
	\closesection
\end{document}