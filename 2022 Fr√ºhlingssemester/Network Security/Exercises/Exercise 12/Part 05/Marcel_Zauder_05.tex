\documentclass{report}

\usepackage{../../../../../LaTeX/marzstyle}

\newcommand{\exercisenr}{12}

\runningheads{Network Security}{Exercise \exercisenr}

\setcounter{chapter}{\exercisenr}
\setcounter{section}{4}


\begin{document}
	\section{Question 5}
	\startsection
		\renewcommand{\thesubsection}{\thesection.\Alph{subsection}}
		\subsection{Onion Routing (\textsc{Group} R3)}
		\startsubsection
			Public Key: e=511, n=851 \\
			Private Key: p=37, q=23, d=31 $\Rightarrow$ n = 37*23 = 851 \\ \\
			Received Message from R2: [699, 759, 306, 537, 225, 277, 277, 815, 437, 432, 47, 47, 288, 288, 288, 460, 301, 487, 277, 460, 144, 232, 47, 319, 599, 336, 404, 587, 313, 516] \\ \\
			Decryption process: $m = c^d \textit{ mod } n$ using a calculator and in the end mapping the results to ASCII characters using the ASCII table. \\ \\
			\noindent Decrypted message [numbers]: [71, 69, 84, 32, 104, 116, 116, 112, 115, 58, 47, 47, 119, 119, 119, 46, 98, 105, 116, 46, 108, 121, 47, 51, 70, 67, 118, 50, 113, 89] \\
			Decrypted message [ascii]: [G, E, T, \_, h, t, t, p, s, :, /, /, w, w, w, ., b, i, t, ., l, y, /, 3, F, C, v, 2, q, Y] \\
			\\
			$\Rightarrow$ <<GET https://www.bit.ly/3FCv2qY>>
		\closesection
		\subsection{The encryption scheme in question 5 A isn’t particularly secure. What are its weaknesses? How would you make it more secure?}
		\startsubsection
			The problem with this encryption scheme is that there is a one-to-one correspondence between the ASCII character and its cipher, meaning that the same cipher corresponse to the same character. Therefore an analysis of the ciphertext can be performed and a mapping of each cipher character to its original plain character can be made. \\
			A solution for this exploit would be to add some kind of randomness that can be reverted like using a seed or a predefined random function in order to mitigate such an analysis attack.
		\closesection
	\closesection
\end{document}