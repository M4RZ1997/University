\documentclass{report}

\usepackage{../../../../../LaTeX/marzstyle}

\newcommand{\exercisenr}{7}

\runningheads{Network Security}{Exercise 0\exercisenr}

\setcounter{chapter}{\exercisenr}
\setcounter{section}{0}


\begin{document}
	\section{Question 1}
	\startsection
		\renewcommand{\thesubsection}{\thesection.\Alph{subsection}}
		\subsection{Match the following terms within the context of Wireless Security.}
		\begin{tabular}{V{2.5}p{0.5cm}|p{3cm}V{2.5}p{0.5cm}|p{10cm}V{2.5}}
			\hlineB{2.5}
			\bfseries 1- & Association: & \bfseries J- & Users may connect to an incorrect network and get or send confidential resources. \\
			\hline
			\bfseries 2- & Ad-hoc networks: & \bfseries G- & Peer-to-peer network without central control, so difficult to manage. \\
			\hline
			\bfseries 3- & Non-traditional networks: & \bfseries L- & Personalized networks and addresses can lead to spoofing or eavesdropping. \\
			\hline
			\bfseries 4- & Identity theft: & \bfseries P- & Getting access to the MAC of privileged devices. \\
			\hline
			\bfseries 5- & Man-in-the-middle attack: & \bfseries F- & Works well in wireless networks since the connections are scattered all around the devices. \\
			\hline
			\bfseries 6- & Denial of service: & \bfseries C- & Works well in wireless networks since it is easy to direct multiple messages to the target. \\
			\hline
			\bfseries 7- & Network injection: & \bfseries N- & Attacks on access points that are exposed to non-filtered network traffic. \\
			\hline
			\bfseries 8- & Signal hiding: & \bfseries A- & Disable SSID broadcasting, change SSID to cryptic value, or reduce signal strength. \\
			\hline
			\bfseries 9- & Encryption: & \bfseries B- & Encrypt the transmissions to avoid eavesdropping. \\
			\hline
			\bfseries 10- & Port-based network control: & \bfseries K- & Only allow traffic on controlled ports and restrict traffic on specific ports. \\
			\hline
			\bfseries 11- & Antivirus: & \bfseries H- & Against malware code inside network (that could open backdoors). \\
			\hline
			\bfseries 12- & Whitelist: & \bfseries E- & Only allow specific computers to the network. \\
			\hline
			\bfseries 13- & Channel: & \bfseries M- & The medium through which the messages are being transmitted. \\
			\hline
			\bfseries 14- & Mobility: & \bfseries I- & Makes the system dynamic leading to various introduced risks. \\
			\hline
			\bfseries 15- & Resources: & \bfseries O- & Are helpful against computational-demanding attacks.\\
			\hline
			\bfseries 16- & Accessibility: & \bfseries D- & Is an important factor because hardware should not be reached without being noticed. \\
			\hlineB{2.5}
		\end{tabular}
	\closesection
\end{document}