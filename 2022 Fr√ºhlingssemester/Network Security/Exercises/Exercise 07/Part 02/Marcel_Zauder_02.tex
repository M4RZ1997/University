\documentclass{report}

\usepackage{../../../../../LaTeX/marzstyle}

\newcommand{\exercisenr}{7}

\runningheads{Network Security}{Exercise 0\exercisenr}

\setcounter{chapter}{\exercisenr}
\setcounter{section}{1}


\begin{document}
	\section{Question 2}
	\startsection
		\renewcommand{\thesubsection}{\thesection.\Alph{subsection}}
		\subsection{Among the possible security threats posed by mobile devices in enterprise networks, which one has the biggest risk-potential and which one the smallest? Why?}
		\startsubsection
			The biggest security risk-potential had the use of applications created by unknown parties, as with insufficient security control these apps can leak for example login and other valuable data. Furthermore, they can come with their own backdoors and other security vulnerabilities. \\
			On the other hand the smalles risk would be the breach of location data, as they are only useful to determine where the mobile device - and his owner - is currently located, as this information is not the valuable in an enterprise network.
		\closesection
		
		\subsection{Can you think of further security strategies for mobile devices, than those already presented?}
		\startsubsection
			A useful security strategy for mobile devices would be to secure the actual device with the use of Hardware Security Modules (HSM) which make sure that the hardwar used within the devices was not tampered with. Especially when using confidential data a data leakage or breach would be fatal, hence making it of most importance to secure those data against any malicious attack.
		\closesection
	\closesection
\end{document}