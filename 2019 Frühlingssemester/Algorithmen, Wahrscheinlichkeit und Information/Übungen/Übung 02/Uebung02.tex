\documentclass{article}
\usepackage{geometry}
\usepackage{paralist}

\usepackage{fancyhdr}
\fancyhead[L]{\LARGE Algorithmen, Wahrscheinlichkeit und Information \\
\Large \"Ubungsserie 02}
\fancyhead[R]{16-124-836 \\
Marcel \textsc{Zauder}}
\renewcommand{\headrulewidth}{0.4pt}
\fancyfoot[C]{\thepage}
\renewcommand{\footrulewidth}{0.4pt}

\usepackage{hyperref}

\begin{document}
\pagestyle{fancy}
\hfill \\

\textbf{\underline{2.1 Zwei W\"urfel}} \\
\\
\underline{Allgemein gilt:} $E[Y \mid B] = \frac{E[1\textsubscript{B} \cdot Y]}{P(B)}$ \\
\begin{compactenum}[a)]
	\item E[Z$\mid$X ist gerade] \\
	\\
	P(X ist gerade) = $\frac{1}{2}$ \\
	E[1\textsubscript{X ist gerade} $\cdot$ Z] = P(Z=1$\mid$X ist gerade)$\cdot$1 + P(Z=2$\mid$X ist gerade)$\cdot$2 +\ldots + P(Z=12$\mid$X ist gerade)$\cdot$12 \\
	E[1\textsubscript{X ist gerade} $\cdot$ Z] = $\frac{1}{36}\cdot$3 + $\frac{1}{36}\cdot$4 + $\frac{2}{36}\cdot$5 + $\frac{2}{36}\cdot$6 + $\frac{3}{36}\cdot$7 + $\frac{3}{36}\cdot$8 + $\frac{2}{36}\cdot$9 + $\frac{2}{36}\cdot$ 10 + $\frac{1}{36}\cdot$11 + $\frac{1}{36}\cdot$12 \\
	= $\frac{135}{36}$ = 3,75 \\
	$\rightarrow$  E[Z$\mid$X ist gerade] = $\frac{3,75}{\frac{1}{2}}$ = 7,5 \\
	\\
	\item E[Z$\mid$Y ist ungerade] \\
	\\
	P(Y ist ungerade) = $\frac{1}{2}$ \\
	E[1\textsubscript{Y ist ungerade} $\cdot$ Z] = P(Z=1$\mid$Y ist ungerade)$\cdot$1 + P(Z=2$\mid$Y ist ungerade)$\cdot$2 +\ldots + P(Z=12$\mid$Y ist ungerade)$\cdot$12 \\
	E[1\textsubscript{Y ist ungerade} $\cdot$ Z] = $\frac{1}{36}\cdot$2 + $\frac{1}{36}\cdot$3 + $\frac{2}{36}\cdot$4 + $\frac{2}{36}\cdot$5 + $\frac{3}{36}\cdot$6 + $\frac{3}{36}\cdot$7 + $\frac{2}{36}\cdot$8 + $\frac{2}{36}\cdot$ 9 + $\frac{1}{36}\cdot$10 + $\frac{1}{36}\cdot$11 \\
	= $\frac{117}{36}$ = 3,25 \\
	$\rightarrow$  E[Z$\mid$Y ist ungerade] = $\frac{3,25}{\frac{1}{2}}$ = 6,5 \\
	\\
	\item E[X$\mid$Z = 5] \\
	\\
	P(Z = 5) = P(X = 1,Y = 4) + P(X = 2,Y = 3) + P(X = 3,Y = 2) + P(X = 4,Y = 1) = $\frac{4}{36}$ = $\frac{1}{9}$ \\
	E[1\textsubscript{Z = 5} $\cdot$ X] = P(X=1$\mid$Z=5)$\cdot$1 + P(X=2$\mid$Z=5)$\cdot$2 +\ldots + P(X=6$\mid$Z=5)$\cdot$6 \\
	E[1\textsubscript{Z = 5} $\cdot$ X] = $\frac{1}{36}$ $\cdot$1 + $\frac{1}{36}$ $\cdot$2 + $\frac{1}{36}$ $\cdot$3 + $\frac{1}{36}$ $\cdot$4 = $\frac{10}{36}$ \\
	$\rightarrow$  E[X$\mid$Z = 5] = $\frac{\frac{10}{36}}{\frac{1}{9}}$ = 2,5 \\
	\\
	\item E[Y$\mid$Z $>$ 6] \\
	\\
	P(Z $>$ 6) = 1 - P(Z $\leq$ 6) = 1 - [P(X = 1,Y = 1) +  P(X = 1,Y = 2) + \ldots + P(X = 5,Y = 1) \\
	= 1 - $\frac{15}{36}$ = $\frac{21}{36}$ \\
	E[1\textsubscript{Z $>$ 6} $\cdot$ Y] = P(Y=1$\mid$Z$>$6)$\cdot$1 + P(Y=2$\mid$Z$>$6)$\cdot$2 +\ldots + P(Y=6$\mid$Z$>$6)$\cdot$6 \\
	E[1\textsubscript{Z $>$ 6} $\cdot$ Y] = $\frac{1}{36}\cdot$1 + $\frac{2}{36}\cdot$2 + $\frac{3}{36}\cdot$3 + $\frac{4}{36}\cdot$4 + $\frac{5}{36}\cdot$5 + $\frac{6}{36}\cdot$6 = $\frac{91}{36}$\\
	$\rightarrow$  E[Y$\mid$Z $>$ 6] = $\frac{\frac{91}{36}}{\frac{21}{36}}$ = $\frac{13}{3}$ = 4,3333333... \\
\end{compactenum}

\newpage

\textbf{\underline{2.2 Jensen im Quadrat}} \\
\\
\underline{zu Beweisen:} E[X\textsuperscript{k}] $\geq$ E[X]\textsuperscript{k} \hspace{1cm} k $\geq$ 2; k $\in$ N \\
\\
\underline{Wir wissen:} Wenn die Funktion f(x) zweimal differenzierbar ist, mit f''(x) $\geq$ 0 $\forall$ x, dann ist sie konvex. \\
Sei nun f(x) := x\textsuperscript{2k} mit k $\in$ N. Dann ist: \\
\[
	f''(x) = k * (k-1) * x^{k-2} \geq 0
\]
Somit folgt, dass f(x) = x\textsuperscript{k} konvex ist für alle geraden k. \\
Durch die Jensen-Ungleichung wissen wir, dass für eine konvexe Funktion f und $\lambda_i > 0$ mit $\sum_{i=1}^n \lambda_i = 1$ gilt: \\
\[
	f(\sum_{i=1}^n{\lambda_i x_i}) \leq \sum_{i=1}^n{\lambda_i f(x_i)}
\]
Sei nun $\lambda_i := P(X = x)$, $f(x) = x^k$:
\[
	E[X]^k = (\sum_{i=1}^n{P(X=x_i) * x_i})^k \\
	= f(\sum_{i=1}^n{\lambda_i * x_i}) \\
	\leq \sum_{i=1}^n{\lambda_i * f(x_i)} \\
	= \sum_{i=1}^n{P(X=x_i) * x_i^k} \\
	= E[X^k]
\]
\\
\\

\textbf{\underline{2.3 Min-Max im Erwartungswert}} \\
\begin{compactenum}[a)]
	\item Was ist E[min(A,B)] und E[max(A,B)]? \\
	\[
	E[max(A, B)] = \sum_{a} \sum_{b} max(a, b) \cdot \frac{1}{k} \cdot \frac{1}{k}
	= \frac{1}{k^2} \sum_{a} \sum_{b \leq a} a + \sum_{b>a} b $$ $$
	E[min(A, B)] = \sum_{a} \sum_{b} min(a, b) \cdot \frac{1}{k} \cdot \frac{1}{k}
	= \frac{1}{k^2} \sum_{a} \sum_{b \leq a} b + \sum_{b>a} a $$ $$
	\]
	\\
	\item \underline{zu Zeigen:} E[min(A,B)] + E[max(A,B)] = E[A] + E[B] \\
	\[
	E[max(A, B)] + E[min(A, B)] = \frac{1}{k^2} \sum_{a} \sum_{b \leq a} a + \sum_{b>a} b + \frac{1}{k^2} \sum_{a} \sum_{b \leq a} b + \sum_{b>a} a $$ $$
	= \frac{1}{k^2} \cdot \sum_{a} \left(\sum_{b} b + \sum_{b} a \right) = \frac{1}{k^2} \cdot \sum_{a}\sum_{b} a+b = E[A] + E[B] $$ $$
	\]
	\newpage
	\item Beweis durch Eigenschaften des Erwartungswertes \\
	\[
	E[max(A, B)] + E[min(A, B)] = E[max(A, B) + min(A, B)] = E[A + B] = E[A] + E[B]
	\]
	\\
	
\end{compactenum}

\textbf{\underline{2.4 Ein zuf\"alliger Text}} \\
\\
weitere Annahmen sind zu treffen, da sonst die Aufgabe nicht zu lösen ist: \\
\begin{compactenum}[(i)]
	\item Die Katze trifft bei jedem Schritt genau ein Zeichen \\
	\item der nächste Buchstabe ist unabhängig von der vorherigen Eingabe (ein Zeichen kann auch doppelt hintereinander eingegeben werden) \\
	\item Wir beachten nicht, dass die Katze auch weniger oder mehr Zeichen eingeben kann, sondern, dass sie immer genau die angegebene Zeichenanzahl erreicht \\
\end{compactenum}
\begin{compactenum}[a)]
	\item Ihr Passwort (10 Zeichen) \\	
	P(Passwort) = $(\frac{1}{32})$\textsuperscript{10} = 8,88 $\cdot$ 10\textsuperscript{-16} \\
	
	\item Den 128-bit AES-Schl\"ussel einer TLS-Verbindung auf dem Internet \\	
	Der AES-Schl\"ussel besteht in diesem Beispiel aus 26 Zeichen, also: \\
	P(Passwort) = $(\frac{1}{32})$\textsuperscript{26} = 7,35 $\cdot$ 10\textsuperscript{-40} \\
	
	\item Die Kopfzeile dieses \"Ubungsblattes (125 Zeichen) \\
	P(Passwort) = $(\frac{1}{32})$\textsuperscript{125} = 7,18 $\cdot$ 10\textsuperscript{-189} \\
	
	\item \textbf{Zusatz:} Computer mit 40 $\cdot$ 10\textsuperscript{18} Operationen pro Sekunde \\
	\begin{compactenum}[(i)]
		\item Passwort \\
			T\textsubscript{max}(Passwort) = $\frac{1}{2} \cdot\frac{32\textsuperscript{10}}{40 \cdot 10\textsuperscript{18}} = 1,41 \cdot 10\textsuperscript{-5}$ sec \\
		\item AES-Schl\"ussel \\
		T\textsubscript{max}(Passwort) = $\frac{1}{2} \cdot\frac{32\textsuperscript{26}}{40 \cdot 10\textsuperscript{18}} = 1,70 \cdot 10\textsuperscript{19}$ sec \\
		\item Kopfzeile \\
		T\textsubscript{max}(Passwort) = $\frac{1}{2} \cdot\frac{32\textsuperscript{125}}{40 \cdot 10\textsuperscript{18}} =  1,74 \cdot 10\textsuperscript{168}$ sec (sehr sehr lange) \\
	\end{compactenum}
\end{compactenum}

\end{document}