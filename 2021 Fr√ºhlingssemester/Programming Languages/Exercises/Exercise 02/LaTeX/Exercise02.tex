\documentclass{article}
\usepackage{geometry}
\usepackage{paralist}
\usepackage[T1]{fontenc}
\usepackage{reledmac}
\usepackage{changepage}
\usepackage{amsmath}
\usepackage{scalerel,amssymb}
\usepackage{colortbl}

\usepackage[most]{tcolorbox}

\usepackage{pgfplots}
\usepackage{tikz}
\usetikzlibrary{positioning}
\usetikzlibrary{shapes.geometric, arrows}
\tikzstyle{arrow} = [thick,->,>=stealth]

\usepackage{fancyhdr}
\fancyhead[L]{
	\begin{tabular}{l}
		\LARGE \textbf{\textsc{Programming Languages}} \\
		\large Exercise 02
	\end{tabular}
}
\fancyhead[R]{
	\begin{tabular}{r}
		16-124-836 \\
		Marcel \textsc{Zauder}
	\end{tabular}
}
\renewcommand{\headrulewidth}{0.4pt}
\fancyfoot[C]{\thepage}
\renewcommand{\footrulewidth}{0.4pt}


\usepackage{hyperref}

\begin{document}
	\pagestyle{fancy}
	\hfill
	
	\subsection*{1.1 PostScript}
	\begin{adjustwidth}{2em}{2em}
		\subsubsection*{1.1.1 What kinds of stacks does PostScript manage and what are their roles?}
		\begin{adjustwidth}{2em}{}
			First we have an \textbf{Operand Stack} which holds (arbitrary) operands and results of PostScript operators. \\
			Then we have a \textbf{Dictionary Stack}, which holds only dictionaries where keys and values may be stores. \\
			An \textbf{Execution Stack} holds executable objects like procedures in stages of the execution. \\
			And last we have the \textbf{Graphics State Stack} that keeps track of current coordinates.
		\end{adjustwidth}
		\subsubsection*{1.1.2 What is the way of defining a procedure in the PostScript program?}
		\begin{adjustwidth}{2em}{}
			Procedures are defined by binding names to literal or executable objects. For example we can create the following line of code:
			\begin{adjustwidth}{2em}{2em}
			\begin{tcolorbox}
				\begin{verbatim}
					/square { dup mul } def
				\end{verbatim}
			\end{tcolorbox}
			\end{adjustwidth}
			\noindent With this we created a procedure square, which first doublicates the top element of the stack and then multiplies both of the elements. Therefore we can write:
			\begin{adjustwidth}{2em}{2em}
			\begin{tcolorbox}
				\begin{verbatim}
					3 square
				\end{verbatim}
			\end{tcolorbox}
			\end{adjustwidth}
			\noindent and get the result 9. \\
			A procedure calculating $((x+y) / 2) \cdot 2$ looks like the following:
			\begin{adjustwidth}{2em}{2em}
			\begin{tcolorbox}
				\begin{verbatim}
					/calculation { add 2 div 2 mul } def
				\end{verbatim}
			\end{tcolorbox}
			\end{adjustwidth}
			\noindent So the whole program can look like the following (example for $x=3$ and $y=4$):
			\begin{adjustwidth}{2em}{2em}
			\begin{tcolorbox}
				\begin{verbatim}
					/procedure { add 2 div 2 mul } def
					/sBuf { 20 string } def
					/showInt { sBuf cvs show } def
					/printCalculation { procedure showInt } def
					
					/Times-Roman findfont 18 scalefont setfont
					100 500 moveto
					3 4 printCalculation
					showpage
				\end{verbatim}
			\end{tcolorbox}
			\end{adjustwidth}
		\end{adjustwidth}
		\subsubsection*{1.1.3 Procedure for printing 10 random numbers on separate lines}
		\begin{adjustwidth}{2em}{}
			\begin{tcolorbox}
				\begin{verbatim}
					/LM 100 def
					/FS 18 def
					/newline {
					    currentpoint exch pop
					    FS 2 add sub
					    LM exch moveto
					} def
					/sBuf { 20 string } def
					/showInt { sBuf cvs show } def
					
					/Times-Roman findfont 18 scalefont setfont
					LM 600 moveto
					1 1 10 { rand showInt newline } for
					showpage
				\end{verbatim}
			\end{tcolorbox}
		\end{adjustwidth}
	\end{adjustwidth}
	
	\subsection*{1.2 Catalan Numbers in PostScript}
	\begin{adjustwidth}{2em}{2em}
	\begin{tcolorbox}
		\begin{verbatim}
			/LM 100 def % left margine
			/UM 700 def % upper margine
			/FS 18 def % font size

			/showNum { 20 string cvs show } def

			/showCatalan {
			    (C \( n = ) show dup showNum ( \) = ) show calculation showNum
			} def

			/factorial{
			    dup 1 lt { pop 1 } { dup 1 sub factorial mul }
			    ifelse
			} def

			/newLine {
			    currentpoint exch pop
			    FS 2 add sub
			    LM exch moveto
			} def

			/catalan { 0 exch 1 exch { showCatalan newLine } for } def

			/calculation {
			    dup nominator exch denominator div
			} def

			/nominator { 2 mul factorial } def

			/denominator { dup nplus1fact exch factorial mul } def

			/nplus1fact { 1 add factorial } def

			/Times-Roman findfont FS scalefont setfont
			%Usage: n catalan
			LM UM moveto
			17 catalan
		\end{verbatim}
	\end{tcolorbox}
	\hfill \\
	\noindent First we create a for loop by adding 0 and 1 in front of $n$ using \textit{exch}. Inside the for loop each \textbf{Catalan-Number} is computed by first calculating the nominator and then the denominator. In the end both are used in a division to get the result. These values are then printed with the \textit{showNum} procedure and each value has its individual line, by using the \textit{newLine} procedure.
	\end{adjustwidth}
\end{document}