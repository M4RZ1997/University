\documentclass{article}
\usepackage{geometry}
\usepackage{paralist}
\usepackage[T1]{fontenc}
\usepackage{reledmac}
\usepackage{changepage}
\usepackage{amsmath}
\usepackage{scalerel,amssymb}
\usepackage{colortbl}

\usepackage[most]{tcolorbox}

\usepackage{pgfplots}
\usepackage{tikz}
\usetikzlibrary{positioning}
\usetikzlibrary{shapes.geometric, arrows}
\tikzstyle{arrow} = [thick,->,>=stealth]

\usepackage{fancyhdr}
\fancyhead[L]{
	\begin{tabular}{l}
		\LARGE \textbf{\textsc{Programming Languages}} \\
		\large Exercise 03
	\end{tabular}
}
\fancyhead[R]{
	\begin{tabular}{r}
		16-124-836 \\
		Marcel \textsc{Zauder}
	\end{tabular}
}
\renewcommand{\headrulewidth}{0.4pt}
\fancyfoot[C]{\thepage}
\renewcommand{\footrulewidth}{0.4pt}


\usepackage{hyperref}

\begin{document}
	\pagestyle{fancy}
	\hfill
	
	\subsection*{3.1 Why does the piece of code not raise an error?}
	\begin{adjustwidth}{2em}{2em}
		Because of using \textit{"lazy load/execution"} in functional programming this code would not raise an error. This is because when defining \textit{funcl y z = y}, z is not needed for executing this computation. Therefore the \textit{sqrt(-5)} is not considered for the output and therefore no error occurs.
	\end{adjustwidth}
	
	\subsection*{3.2 Small Program Definition}
	\begin{adjustwidth}{2em}{2em}
		\begin{tcolorbox}
			\begin{verbatim}
				if n = 0 then
				    return -1
				else
				    return n*2
			\end{verbatim}
		\end{tcolorbox}
		\subsubsection*{3.2.1 Pattern Matching}
		\begin{adjustwidth}{2em}{}
			\begin{tcolorbox}
				\begin{verbatim}
					pattern 0 = -1
					pattern n = n*2
				\end{verbatim}
			\end{tcolorbox}
		\end{adjustwidth}
		\subsubsection*{3.2.2 Guards}
		\begin{adjustwidth}{2em}{}
			\begin{tcolorbox}
				\begin{verbatim}
					guards n | n == 0  = -1
					           | n >= 1  = n*2
				\end{verbatim}
			\end{tcolorbox}
		\end{adjustwidth}
		\subsubsection*{3.2.3 Lambda Expression}
		\begin{adjustwidth}{2em}{}
			\begin{tcolorbox}
				\begin{verbatim}
					lambda = ( \n -> if n == 0 then -1 
					          else n*2 )
				\end{verbatim}
			\end{tcolorbox}
		\end{adjustwidth}
	\end{adjustwidth}
	
	\subsection*{3.3 Program that computes the sum of all members of a list}
	\begin{adjustwidth}{2em}{2em}
		\begin{tcolorbox}
			\begin{verbatim}
				listsum[] = 0
				listsum(x:xs) = x + listsum xs
				
				main = print $ listsum[1,2,3,4,5,6,7,8,9]
			\end{verbatim}
		\end{tcolorbox}		
	\end{adjustwidth}
	
	\subsection*{3.4 Catalan Numbers in Haskell}
	\begin{adjustwidth}{2em}{2em}
	\begin{tcolorbox}
		\begin{verbatim}
			fac 0 = 1
			fac n = n * fac (n-1)
			
			getList n = [0..n]
			catalan n = fac (2*n) / (fac(n+1) * fac(n))
			
			firstNcatalan n = map catalan (getList(n-1))
			
			main = print $ firstNcatalan 17
		\end{verbatim}
	\end{tcolorbox}
	\end{adjustwidth}
\end{document}