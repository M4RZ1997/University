\documentclass{article}
\usepackage{geometry}
\usepackage{paralist}
\usepackage[T1]{fontenc}
\usepackage{reledmac}
\usepackage{changepage}
\usepackage{amsmath}
\usepackage{scalerel,amssymb}
\usepackage{colortbl}

\usepackage[most]{tcolorbox}

\usepackage{pgfplots}
\usepackage{tikz}
\usetikzlibrary{positioning}
\usetikzlibrary{shapes.geometric, arrows}
\tikzstyle{arrow} = [thick,->,>=stealth]

\usepackage{fancyhdr}
\fancyhead[L]{
	\begin{tabular}{l}
		\LARGE \textbf{\textsc{Programming Languages}} \\
		\large Exercise 04
	\end{tabular}
}
\fancyhead[R]{
	\begin{tabular}{r}
		16-124-836 \\
		Marcel \textsc{Zauder}
	\end{tabular}
}
\renewcommand{\headrulewidth}{0.4pt}
\fancyfoot[C]{\thepage}
\renewcommand{\footrulewidth}{0.4pt}


\usepackage{hyperref}

\begin{document}
	\pagestyle{fancy}
	\hfill
	
	\subsection*{4.1 Monomorphic or Polymorphic Functions}
	\begin{adjustwidth}{2em}{2em}
		\begin{tcolorbox}
			\begin{verbatim}
				mod :: Int -> Int -> Int
				factors n = [x \mid x <- [1..n-1], mod n x == 0 ]
			\end{verbatim}
		\end{tcolorbox}
		\noindent The function factors is monomorphic because it only accepts inputs of type \texttt{Int} and always will return a result of type \texttt{[Int]}. \\
		\texttt{factors :: Int -> [Int]} \\
		\begin{tcolorbox}
			\begin{verbatim}
				isPerfect = sum (factors n) == n
			\end{verbatim}
		\end{tcolorbox}
		\noindent The function isPerfect is monomorphic because it only accepts inputs of type \texttt{Int} and always will return a result of type \texttt{Boolean}. \\
		\texttt{isPerfect :: Int -> Boolean}
		\begin{tcolorbox}
			\begin{verbatim}
				insert _ n [] = [n]
				insert 0 n l = n:l
				insert i n (x:xs) = x : insert (i-1) n xs
			\end{verbatim}
		\end{tcolorbox}
		\noindent The function insert n is polymorphic because the type of input for n could be either one element or a list. \\
		\texttt{insert :: Int -> a -> [a] -> [a]}
		\begin{tcolorbox}
			\begin{verbatim}
				mH (a, b, c) = c
			\end{verbatim}
		\end{tcolorbox}
		\noindent The function mH is polymorphic because the type of input and output can vary. \\
		\texttt{mH :: (a, b, c) -> c}
	\end{adjustwidth}
	
	\subsection*{4.2 Square Function}
	\begin{adjustwidth}{2em}{2em}
		\begin{tcolorbox}
			\begin{verbatim}
				square :: Int -> Int
				square :: Float -> Float
				square :: Char -> Char
				square :: Double -> Double
			\end{verbatim}
		\end{tcolorbox}
		\noindent Because a square of a char is not defined in a sensible way, this type of input and output would be invalid.
	\end{adjustwidth}
	
	\subsection*{4.3 Function to calculate Circumference of a Circle and Area of a Rectangle}
	\begin{adjustwidth}{2em}{2em}	
		\begin{tcolorbox}
			\begin{verbatim}
				data Shape = Circle Float | Rectangle Float Float
				
				--exercise :: Shape -> Float
				exercise (Circle a) =  2 * 3.14 * a
				exercise (Rectangle a b) = a * b
			\end{verbatim}
		\end{tcolorbox}	
	\end{adjustwidth}
\end{document}