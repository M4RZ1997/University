\documentclass{article}
\usepackage{geometry}
\usepackage{paralist}
\usepackage[T1]{fontenc}
\usepackage{reledmac}
\usepackage{changepage}
\usepackage{amsmath}
\usepackage{scalerel,amssymb}
\usepackage{colortbl}

\usepackage{pgfplots}
\usepackage{tikz}
\usetikzlibrary{positioning}
\usetikzlibrary{shapes.geometric, arrows}
\tikzstyle{arrow} = [thick,->,>=stealth]

\usepackage{fancyhdr}
\fancyhead[L]{
	\begin{tabular}{l}
		\LARGE \textbf{\textsc{Cryptographic Protocols}} \\
		\large Exercise 01
	\end{tabular}
}
\fancyhead[R]{
	\begin{tabular}{r}
		16-124-836 \\
		Marcel \textsc{Zauder}
	\end{tabular}
}
\renewcommand{\headrulewidth}{0.4pt}
\fancyfoot[C]{\thepage}
\renewcommand{\footrulewidth}{0.4pt}

\usepackage{hyperref}

\begin{document}
	\pagestyle{fancy}
	\hfill
	
	\section*{1.1 Calculation in finite fields}
	\begin{adjustwidth}{2em}{2em}
		\textit{Evaluate the following polynomials:}
		\subsection*{1.1.1 Evaluate the polynomial $r(X)$}
		\begin{align*}
			r(X) \ & = \ 3 \cdot X^3 + 2 \cdot X^2 + X & & \in GF(5)[X] \textit{ at } X = 2 \\
			r(X) \ & = \ (3 \cdot 2^3 + 2 \cdot 2^2 + 2) \textit{ mod } 5 & & \textit{ because } p = 5 \textit{ is prime} \\
			& = \ (24 + 8 + 2) \textit{ mod } 5 \\
			& = \ 34 \textit{ mod } 5 \\
			& = \ 4				
		\end{align*}
		\subsection*{1.1.2 Evaluate the polynomial $s(X)$}
		\begin{align*}
			s(X) \ & = \ (1+\alpha) \cdot X^3 + \alpha \cdot X^2 + X & & \in GF(4)[X] \textit{ at } X = \alpha \\
			s(X) \ & = \ (1+\alpha) \cdot \alpha^3 + \alpha \cdot \alpha^2 + \alpha \\
			& = \ (1+\alpha) \cdot \alpha \cdot \alpha \cdot \alpha + \alpha \cdot \alpha \cdot \alpha + \alpha \\
			& = \ 1 \cdot \alpha \cdot \alpha + (1 + \alpha) \cdot \alpha + \alpha & & \textit{,because } (1+\alpha) \cdot \alpha = 1 \textit{ and } \alpha \cdot \alpha = 1 + \alpha \\
			& = \ \alpha \cdot \alpha + 1 + \alpha & & \textit{,because } 1 \cdot \alpha = \alpha \textit{ and } (1 + \alpha) \cdot \alpha = 1 \\
			& = \ (1 + \alpha) + 1 + \alpha & & \textit{,because } \alpha \cdot \alpha = (1+\alpha) \\
			& = \ \alpha + \alpha & & \textit{,because } (1+\alpha) + 1 = \alpha \\
			& = \ 0
		\end{align*}
	\end{adjustwidth}
	
	\section*{1.2 Trivial functions for secure computation}
	\begin{adjustwidth}{2em}{2em}
		\begin{enumerate}[a)]
			\item is trivial
			\item is non-trivial, because if the value of $x$ is greater than the value of $y$, e.g. $x = 5$ in $\mathbb{G}\mathbb{F}(5)$, we cannot determine whether $y$ was, e.g. 2 or 3.
			\item is non-trivial, because if the value of $x$ is 0, then no matter which value was assigned to $y$, there is no way to determine that value after the computation, because $f$ will return 0.
			\item is trivial
			\item is non-trivial, because there can be more than one value which is greater or less than the chosen value $x$ in the finite field $\mathbb{G}\mathbb{F}(5)$, e.g. $x=3$ then $y$ could be 2 or 1 to produce the output 1.
		\end{enumerate}
	\end{adjustwidth}
	
	\section*{1.3 Non-trivial functions and an embedded OR}
	\begin{adjustwidth}{2em}{2em}
		\begin{enumerate}
			\item[b)] The corresponding table looks as follows: \\
			\begin{tabular}{| c | c | c | c | c | c |}
				\hline
				$max(x, y)$ & 0 & 1 & 2 & 3 & 4 \\
				\hline
				0 & \cellcolor[gray]{0.8}0 & \cellcolor[gray]{0.8}1 & \cellcolor[gray]{0.8}2 & \cellcolor[gray]{0.8}3 & 4 \\
				1 & \cellcolor[gray]{0.8}1 & \cellcolor[gray]{0.8}1 & \cellcolor[gray]{0.8}2 & \cellcolor[gray]{0.8}3 & 4 \\
				2 & \cellcolor[gray]{0.8}2 & \cellcolor[gray]{0.8}2 & \cellcolor[gray]{0.8}2 & \cellcolor[gray]{0.8}3 & 4 \\
				3 & \cellcolor[gray]{0.8}3 & \cellcolor[gray]{0.8}3 & \cellcolor[gray]{0.8}3 & \cellcolor[gray]{0.8}3 & 4 \\
				4 & \cellcolor[gray]{0.8}4 & \cellcolor[gray]{0.8}4 & \cellcolor[gray]{0.8}4 & \cellcolor[gray]{0.8}4 & 4 \\
				\hline
			\end{tabular} \\
			As one can see, when the value 4 is picked for $x$ we cannot determine whether y was e.g. 2 or 3, because the output will always be 4.
			\item[c)] The corresponding table looks as follows: \\
			\begin{tabular}{| c | c | c | c | c | c |}
				\hline
				$x \cdot y$ & 0 & 1 & 2 & 3 & 4 \\
				\hline
				0 & 0 & \cellcolor[gray]{0.8}0 & \cellcolor[gray]{0.8}0 & \cellcolor[gray]{0.8}0 & \cellcolor[gray]{0.8}0 \\
				1 & 0 & \cellcolor[gray]{0.8}1 & \cellcolor[gray]{0.8}2 & \cellcolor[gray]{0.8}3 & \cellcolor[gray]{0.8}4 \\
				2 & 0 & \cellcolor[gray]{0.8}2 & \cellcolor[gray]{0.8}4 & \cellcolor[gray]{0.8}1 & \cellcolor[gray]{0.8}3 \\
				3 & 0 & \cellcolor[gray]{0.8}3 & \cellcolor[gray]{0.8}1 & \cellcolor[gray]{0.8}4 & \cellcolor[gray]{0.8}2 \\
				4 & 0 & \cellcolor[gray]{0.8}4 & \cellcolor[gray]{0.8}3 & \cellcolor[gray]{0.8}2 & \cellcolor[gray]{0.8}1 \\
				\hline
			\end{tabular} \\
			As one can see, when the value 0 is picked for $x$ we cannot determine whether y was e.g. 2 or 3, because the output will always be 0.
			\item[e)] The corresponding table looks as follows: \\
			\begin{tabular}{| c | c | c | c | c | c |}
				\hline
				$e(x,y)$ & 0 & 1 & 2 & 3 & 4 \\
				\hline
				0 & \cellcolor[gray]{0.8}1 & \cellcolor[gray]{0.8}1 & 1 & \cellcolor[gray]{0.8}1 & \cellcolor[gray]{0.8}1 \\
				1 & \cellcolor[gray]{0.8}0 & \cellcolor[gray]{0.8}1 & 1 & \cellcolor[gray]{0.8}1 & \cellcolor[gray]{0.8}1 \\
				2 & \cellcolor[gray]{0.8}0 & \cellcolor[gray]{0.8}0 & 1 & \cellcolor[gray]{0.8}1 & \cellcolor[gray]{0.8}1 \\
				3 & \cellcolor[gray]{0.8}0 & \cellcolor[gray]{0.8}0 & 0 & \cellcolor[gray]{0.8}1 & \cellcolor[gray]{0.8}1 \\
				4 & \cellcolor[gray]{0.8}0 & \cellcolor[gray]{0.8}0 & 0 & \cellcolor[gray]{0.8}0 & \cellcolor[gray]{0.8}1 \\
				\hline
			\end{tabular} \\
			As one can see, when the value 2 is picked for $x$ and the output of the function is 1, we cannot determine whether the value $y$ was 0 or 1 (or even 2).
		\end{enumerate}
	\end{adjustwidth}
\end{document}