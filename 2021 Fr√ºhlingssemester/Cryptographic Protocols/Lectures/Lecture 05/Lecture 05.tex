\documentclass{report}
\usepackage{E:/Documents/GitHub/University/LaTeX/marzstyle}

\setcounter{chapter}{5}

\runningheads{Cryptographic Protocols}{Lecture 05: Zero Knowledge Proofs}

\begin{document}
 
	\section{Analysis}
	\startsection
		\begin{enumerate}
			\item Completeness
			\item Soundness
			\item Zero-Knowledge
		\end{enumerate}
		\subsection{Completeness}
		\startsubsection
			If $\mathbb{G}_0 \tilde{=} \mathbb{G}_1$ then $\mathbb{V}$ always accepts
		\closesection
		\subsection{Soundness}
		\startsubsection
			If $\mathbb{G}_0 \tilde{\neq} \mathbb{G}_1$ then $\mathbb{V}$ rejects with probability at least $\frac{1}{2}$. (But cheating prover may also succeed with probability at most $\frac{1}{2}$. This is called the \textit{soundness error} of the scheme. To reduce it to $2^{-k}$, then repeat the protocol $k$ times.)
		\closesection
		\subsection{Intuition here}
		\startsubsection
			\begin{enumerate}[-]
				\item If $\mathbb{P}$ could answer both challenges, then statement must be true!
				\item If model both as machines (Turing Machines or VMs), then one can take a snapshot of both after $\mathbb{P}$ has sent first message. Then run protocol to completion; Then restart protocol from snapshot and run it until $\mathbb{V}$ picks a different challenge (b). \\
				This gives $\mathbb{V}$ both permutations $\rho _0$ and $\rho _1$ \\
				$\mathbb{G}_0 \stackrel{\tilde{=}}{\rho _0} \mathbb{H}$ and $\mathbb{G}_1 \stackrel{\tilde{=}}{\rho _1} \mathbb{H}$ \\
				so, $\mathbb{V}$ could extract the isomorphism between $\mathbb{G}_0$ and $\mathbb{G}_1$
				\item Gedankenexperiment
			\end{enumerate}
		\closesection
		\subsection{Zero-Knowledge}
		\startsubsection
			What is "computational knowledge"? \\
			If a party can generate a random variable $T$ with exactly the same (or an indistinguishable) distribution, then this party gains no (useful) information from $T$. \\
			Here, $\mathbb{V}$ can generate (simulate) a transcript $T$ of an accepting protocol execution. \\
			same distribution $\Leftrightarrow$ perfect zero knowledge \\
			indistinguishable distribution $\Leftrightarrow$ computational zero knowledge. \\
			$\mathbb{V}$ can simulate transcript $T$:
			\begin{enumerate}
				\item $\tilde{b} \leftarrow \{ 0,1 \}$
				\item $\tilde{\rho} \leftarrow$ permutation of $\mathbb{V}$
				\item $\tilde{H} = \tilde{\rho} (\mathbb{G}_{\tilde{b}})$, i.e. isomorph to $\mathbb{G}_{\tilde{b}}$
			\end{enumerate}
			Distribution ($H, b, \rho$) in real protocol same as ($\tilde{H}, \tilde{b}, \tilde{\rho}$) in simulated execution.
			\begin{enumerate}[$\Rightarrow$]
				\item $\mathbb{V}$ leaves no information through ZKP
				\item $\mathbb{V} $ cannot transfer this to any third party
			\end{enumerate}
		\closesection
		\subsection{What can be proved "in zero-knowledge"?}
		\startsubsection
			\begin{enumerate}[-]
				\item GI problem $\not\in$ P \\
				\indent GI is believed to be between P and NP (like factoring, or DL)
				\item If one NP-complete problem has a ZKP, then any problem in NP has a ZKP (polynomial time) \\
				\indent 3-Colorability of a graph $\mathbb{G}$ is NP-complete and has ZKP
			\end{enumerate}
		\closesection
		\subsection{Can one use these protocols for online authentication?}
		\startsubsection
			In principle yes, BUT in practice more efficient schemes exist.
		\closesection
	\closesection
	
	\section{Zero-Knowledge Proofs of Knowledge (ZKPK)}
	\startsection
		Want to prove knowledge about secrets $\alpha, \beta, \gamma, ...$ such that $\Psi (\alpha, \beta, \gamma, ...)$ holds
		\subsubsection{Example: Prover $\mathbb{P}$}
		\startsubsection
			Prover $\mathbb{P}$ knows that it knows $\alpha$ s.t. $g^{\alpha} = y$.
		\closesection
		\subsubsection{Notation}
		\startsubsection
			$PK \{ (\alpha, \beta, \gamma, ...): \ \Psi (\alpha, \beta, \gamma, ...)\}$, where $\alpha$, $\beta$ are known to $\mathbb{P}$ e.g. $PR\{ (\alpha): \ y = g^{\alpha} \} $
		\closesection
		\subsection{Formalizing ZKPK (3-Move Protocol or $\Sigma$-Protocol)}
		\startsection
			\begin{tikzpicture}
			\end{tikzpicture}
			To convince $\mathbb{V}$, $\mathbb{P}$ should demonstrate that it knows such a secret $(\alpha, \beta, \gamma, ...)$ s.t. $\ \Psi (\alpha, \beta, \gamma, ...)$. \\
			Formalized using \textit{extraction} of the secret $\alpha, \beta, ...$ from $\mathbb{P}$. Using an extractor $\mathbb{E}$, an efficient algorithm that extracts secrets $\alpha, \beta, ...$ from $\mathbb{P}$ when given two protocol runs (transcripts) with same commitment
		\closesection
		\subsection{Definition}
		\startsubsection
			A zero-knowledge proof-of-knowledge (ZKPK) is a 3-Move protocol for a relation $\Psi$ satisfies
			\subsubsection{Completeness}
			\startsubsection
				If $\mathbb{P}$ has input $x$ s.t. $\Psi (x)$ then $\mathbb{V}$ accepts.
			\closesection
			\subsubsection{Soundness}
			\startsubsection
				There is an efficient knowledge extractor $\mathbb{E}$ s.t. $\mathbb{E}((t,c,s), (t,c',s')) \rightarrow x$ when $c \neq c'$ and $\Psi (x)$ (both transcripts are from executions where $\mathbb{V}$ accepts).
			\closesection
			\subsubsection{Zero-Knowledge}
			\startsubsection
				$\mathbb{V}$ can simulate transcripts $(t,c,s)$ on its own with same (or indistinguishable) distribution \\
				$\Leftrightarrow$ \\
				$\exists$ simulator $\mathbb{S}$ that produces $(t,c,s)$... but may use different order
			\closesection
		\closesection
		\subsection{ZKPK of a Discrete Logarithm ("Schnorr Proof")}
		\startsubsection
			Again, $G = <g>$ of order $q$.
			\[
				\Psi (x): g^x = y
			\]
			\begin{tikzpicture}
			\end{tikzpicture}
			\subsubsection{Completeness?}
			\startsubsection
				If $\mathbb{P}$ and $\mathbb{V}$ honest then:
				\[
					t \ = \ g^r \ = \ g^{r-cx+cx} \ = \ g^s \cdot g^{x \cdot c} \ = \ g^s \cdot y^c
				\]
				Thus $\mathbb{V}$ accepts.
			\closesection
			\subsubsection{Soundness?}
			\startsubsection
				Two executions with $(t,c,s)$ and $(t,c',s')$ (Note $c \neq c'$):
				\begin{align*}
					\Rightarrow & \ t = g^s \cdot y^c = g^{s'} \cdot y^{c'} \\
					\Leftrightarrow & \ g^{s-s'} = y^{c'-c} = g^{x \cdot (c' - c)} \\
					\Leftrightarrow & \ s - s' \equiv x (c' - c) \ (\textit{mod } q) \\
					\Leftrightarrow & \ x \equiv \frac{s-s'}{c'-c} \ (\textit{mod } q)
				\end{align*}
				This $x$ satisfies $g^x = y$.
			\closesection
			\subsubsection{Zero-Knowledge?}
			\startsubsection
				$\mathbb{V}$ chooses triples $(t,c,s)$ on its own:
				\begin{align*}
					c & \leftarrow \mathbb{Z}_q \\
					s & \leftarrow \mathbb{Z}_q \\
					t & \leftarrow g^s \cdot y^c \indent (\textit{in G})\\
				\end{align*}
				A triple $(t,c,s)$ has same distribution as a transcript of an accepting execution
			\closesection
		\closesection
	\closesection
	
	\section{Commitment Schemes}
	\startsection
		\begin{enumerate}[-]
			\item How to pick a uniformly random bit among two parties s.t. no single party can bias this bit
			\item Cryptographic primitive for a sender $\mathbb{S}$ and a receiver $\mathbb{R}$
		\end{enumerate}
		\subsection{Definition}
		\startsubsection
			A commitment scheme has 3 algorithms: \textsc{KeyGen()}, \textsc{Com()}, \textsc{Ver()}.
			\begin{enumerate}
				\item $\textsc{KeyGen()} \rightarrow pk$ \\
				\indent probabilistic
				\item $\textsc{Com(pk, x, r)} \rightarrow c$ \\
				\indent deterministic \\
				\indent outputs commitment $c \in \{ 0,1 \} ^{\star}$ \\
				\indent $x \in \{ 0,1 \} ^{\star}$ \\
				\indent $r \in R$, randomness, chosen $r \leftarrow R$ by $\mathbb{S}$
				\item $\textsc{Ver(pk, x, r, c)} \rightarrow \ \textsc{true/false}$ \\
				\indent deterministic, outputs boolean indicating whether $x$ and $r$ correctly "open" commitment $c$ \\
				\indent Run by receiver $\mathbb{R}$
			\end{enumerate}
			\subsubsection{Completeness}
			\startsubsection
				...
			\closesection
			\subsubsection{Binding}
			\startsubsection
				...
			\closesection
			\subsubsection{Hiding}
			\startsubsection
				...
			\closesection
		\closesection
	\closesection

\end{document}