\documentclass{report}
\usepackage{E:/Documents/GitHub/University/LaTeX/marzstyle}

\setcounter{chapter}{6}

\runningheads{Internet of Things}{Exercise 06}

\begin{document}
	\section{Minimum Connected Dominating Sets based on MPR}
	\startsection
		\textbf{Graph:} \\
		\startsection
			\begin{tikzpicture}
				\begin{scope}[every node/.style={circle,thick,draw}]
    				\node[fill=MarzRed] (A) at (0,0) {a};
    				\node[fill=MarzRoyal] (B) at (2,0) {b};
    				\node[fill=MarzRoyal] (C) at (1,-2) {c};
    				\node[fill=MarzGreen] (D) at (3,-2) {d};
    				\node[fill=MarzRoyal] (E) at (2,-4) {e};
    				\node[fill=MarzRoyal] (F) at (4,0) {f} ;
    				\node[fill=MarzRoyal] (G) at (5,-2) {g} ;
    				\node[fill=MarzRed] (H) at (6,0) {h} ;
    				\node[fill=MarzRoyal] (I) at (4,-4) {i} ;
    				\node[fill=MarzRed] (J) at (6,-4) {j} ;
    				\node[fill=MarzRed] (K) at (7,-2) {k} ;
				\end{scope}
				\begin{scope}
    				\path [-] (A) edge (B);
    				\path [-] (A) edge (C);
    				\path [-] (B) edge (C);
    				\path [-] (B) edge (F);
    				\path [-] (F) edge (H);
    				\path [-] (C) edge (D);
    				\path [-] (D) edge (B);
    				\path [-] (D) edge (F);
    				\path [-] (D) edge (G);
    				\path [-] (F) edge (G);
    				\path [-] (G) edge (H);
    				\path [-] (G) edge (K);
    				\path [-] (H) edge (K);
    				\path [-] (E) edge (C);
    				\path [-] (E) edge (D);
    				\path [-] (E) edge (I);
    				\path [-] (D) edge (I);
    				\path [-] (G) edge (I);
    				\path [-] (J) edge (I);
    				\path [-] (J) edge (G);
    				\path [-] (J) edge (K);
				\end{scope}
			\end{tikzpicture}
		\closesection
		1-hop neighbour list of $d$:  $L1 \ = \ [b,c,e,f,g,i]$ \\
		2-hop neighbour list of $d$: $L2 \ = \ [a,h,j,k]$ \\
		\\
		1. Because $g$ is the only node which has a direct connection to the node $k$, it will be put into the set. \\
		2. Remove nodes $h,k,j$ from $L2$ because they are now connected: $L2 \ = \ [a]$ \\
		3. Calculate $\delta(y)$ for each node in $L1$:
		\startsection
			$\delta (b) \ = \ \mid [a] \mid \ = \ 1$ \\
			$\delta (c) \ = \ \mid [a] \mid \ = \ 1$ \\
			$\delta (e) \ = \ \mid [] \mid \ = \ 0$ \\
			$\delta (f) \ = \ \mid [] \mid \ = \ 0$ \\
			$\delta (g) \ = \ \mid [] \mid \ = \ 0$ \\
			$\delta (i) \ = \ \mid [] \mid \ = \ 0$
		\closesection
		4. Select node $y$ with highest $\delta(y)$: Node $b$ \textsc{or} $c$ \\
		5. Remove node $a$ from $L2$ because it is now connected $\Rightarrow$ each 2-hop node is connected and the algorithm terminates. \\
		Therefore we have the \textbf{two} \textit{Minimum CDS} of $d$:
		\startsection
			$[d,g,b]$, if node $b$ has a higher remaining energy value than $c$. \\
			$[d,g,c]$, if node $c$ has a higher remaining energy value than $b$.
		\closesection
	\closesection
	
	\section{Why is Coverage Control in WSNs essential?}
	\startsection
		Coverage Control is essential because the activation of too many sensors and therefore the huge generation of data can congest and overload the network and therefore can be counter-productive. Also energy can be "wasted" by using unnecessary sensors or forwarding of data packets. So it should be always checked that the sensing range $s$ is always greater than the transmission range $r$, then the active routers are sufficient and no new sensors must be introduced. In the end the main goal of coverage control is to preserve the coverage within an area.
	\closesection
\end{document}