\documentclass{article}
\usepackage{geometry}
\usepackage{paralist}
\usepackage[T1]{fontenc}
\usepackage{reledmac}
\usepackage{changepage}
\usepackage{amsmath}

\usepackage{colortbl}

\usepackage{pgfplots}
\usepackage{tikz}
\usetikzlibrary{positioning}
\usetikzlibrary{shapes.geometric, arrows}
\tikzstyle{arrow} = [thick,->,>=stealth]

\usepackage{fancyhdr}
\fancyhead[L]{
	\begin{tabular}{l}
		\LARGE \textbf{\textsc{Internet of Things}} \\
		\large Exercise 03
	\end{tabular}
}
\fancyhead[R]{
	\begin{tabular}{r}
		16-124-836 \\
		Marcel \textsc{Zauder}
	\end{tabular}
}
\renewcommand{\headrulewidth}{0.4pt}
\fancyfoot[C]{\thepage}
\renewcommand{\footrulewidth}{0.4pt}

\usepackage{hyperref}

\begin{document}
	\pagestyle{fancy}
	\hfill
	
	\section*{3.1 What is the difference between event-driven and multi-threading concurrency}
	\begin{adjustwidth}{2em}{2em}
		When using the event-driven concurrency approach it can happen that one single handler can take a lot of time to complete/terminate, which can cause to a "traffic jam" for other requests will not be handled for a long time. In multi-threading concurrency this is not the case because other threads are handled independently from each other and requests will be handled regardless another request is taken a very long time to complete. \\
		Furthermore event-driven concurrency has a very inexpensive memory cost, whereas the memory cost of multi-thread concurrency is much higher.
	\end{adjustwidth}
	
	\section*{3.2 Low-Energy Earliest Deadline First}
	\begin{adjustwidth}{2em}{2em}
		\begin{tabular}{|l|l|l|l|}
			\hline
			\rowcolor{gray!80} Task & Arrival Time [s] & Deadline [s] & Length [MI] \\
			\hline
			t1 & 0 & 4 & 900 \\
			t2 & 2 & 8 & 1800 \\
			t3 & 2.5 & 8 & 450 \\
			t4 & 5 & 20 & 1000 \\
			t5 & 6 & 14 & 800 \\
			\hline
			\rowcolor{gray!80} \multicolumn{2}{|l|}{Processor Speed [MIPS]} & \multicolumn{2}{|l|}{Voltage} \\
			\hline
			\multicolumn{2}{|l|}{200} & \multicolumn{2}{|l|}{1.5} \\
			\multicolumn{2}{|l|}{300} & \multicolumn{2}{|l|}{2} \\
			\multicolumn{2}{|l|}{450} & \multicolumn{2}{|l|}{3.5} \\
			\hline
		\end{tabular}
	\end{adjustwidth}
	\begin{adjustwidth}{-8.5em}{}
		\begin{align*}
			t_0 = 0: & \\
			& 0 + \frac{900}{200} \not\leq 4 \ \Rightarrow \ \textsc{false} \\
			& 0 + \frac{900}{300} \leq 4 \ \Rightarrow \ t = 3 \Rightarrow t1 \textit{ scheduled on } 2V  \\
			t_1 = 2: & \\
			& 2 + \frac{300}{200} \leq 4 \Rightarrow \ t = 3.5 \Rightarrow \exists \tau_2 \textit{ with 1800 MI: } 3.5 + \frac{1800}{450} = 7.5 \leq 8  \Rightarrow t1 \textit{ scheduled on } 1.5V \\
			t_2 = 2.5: & \\
			& 2.5 + \frac{200}{200} \leq 4 \Rightarrow \ t = 3.5 \Rightarrow \exists \tau_2 \textit{ with 1800 MI: } 3.5 + \frac{1800}{450} = 7.5 \leq 8  \Rightarrow \exists \tau_3 \textit{ with 450 MI: } 7.5 + \frac{450}{450} = 8.5 \not\leq 8 \Rightarrow \textsc{break}\\
			& 2.5 + \frac{200}{300} \leq 4 \Rightarrow \ t = 3.1\overline{6} \Rightarrow \exists \tau_2 \textit{ with 1800 MI: } 3.1\overline{6} + \frac{1800}{450} = 7.1\overline{6} \leq 8  \Rightarrow \exists \tau_3 \textit{ with 450 MI: } 7.1\overline{6} + \frac{450}{450} = 8.1\overline{6} \not\leq 8 \Rightarrow \textsc{break}\\
			& 2.5 + \frac{200}{450} \leq 4 \Rightarrow \ t = 2.9\overline{4} \Rightarrow \exists \tau_2 \textit{ with 1800 MI: } 2.9\overline{4} + \frac{1800}{450} = 6.9\overline{4} \leq 8  \Rightarrow \exists \tau_3 \textit{ with 450 MI: } 6.9\overline{4} + \frac{450}{450} = 7.9\overline{4} \leq 8 \Rightarrow \textsc{ok}\\
		\end{align*}
		\begin{align*}
		t_3 = 5: & \\
			& 7.9\overline{4} + \frac{1000}{450} \leq 20 \Rightarrow \textsc{ok} \\
			t_4 = 6: & \\
			& 7.9\overline{4} + \frac{800}{450} = 9.7\overline{1} \leq 14 \Rightarrow 9.7\overline{1} + \frac{1000}{450} \leq 20 \Rightarrow \textsc{ok} \\
			t_6 = 7.9\overline{4}: & \\
			& 7.9\overline{4} + \frac{800}{200} = 11.9\overline{4} \leq 14 \ \Rightarrow \exists \tau_5 \textit{ with 1000 MI: } 11.9\overline{4} + \frac{1000}{450} = 14.1\overline{6} \leq 20 \Rightarrow \textsc{ok} \\
			t_5 = 11.9\overline{4}: & \\
			& 11.9\overline{4} + \frac{1000}{200} = 16.9\overline{4} \leq 20 \Rightarrow t5 \textit{ scheduled on } 1.5V
		\end{align*}
	\end{adjustwidth}
	\begin{adjustwidth}{2em}{2em}
		Therefore we get the execution plan: \\
		\begin{tabular}{|l|l|l|}
			\hline
			\rowcolor{gray!80} Time[s] & Processor Speed [MIPS] & Voltage \\
			\hline
			0-2 & 300 & 2V \\
			2-2.5 & 200 & 1.5V \\
			2.5-7.9$\overline{4}$ & 450 & 3.5V \\
			7.9$\overline{4}$-16.9$\overline{4}$ & 200 & 1.5V \\
			\hline
		\end{tabular}
	\end{adjustwidth}
\end{document}