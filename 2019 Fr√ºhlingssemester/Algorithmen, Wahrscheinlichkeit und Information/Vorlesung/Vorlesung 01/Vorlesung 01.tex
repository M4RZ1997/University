 \documentclass{article}
 \usepackage{hyperref}
 \usepackage[hoffset = 1px]{geometry}
 \begin{document}
 \begin{center}
 \textbf{\underline{Algorithmen, Wahrscheinlichkeit und Information}}
 \end{center}
 \hfill \\ \\ 	
 \textbf{Kapitel 01: Ereignisse und Wahrscheinlichkeit (WSK)}
 \\ \\
 \underline{Def.:} Wahrscheinlichkeitsraum \\
 1) Ergebnisraum $\Omega$ \\
 = Menge aller m\"oglichen Ereignisse \\
 $\omega\in\Omega$: Elementarereignis  \\
 2) Ereignissystem $\Sigma$ \\
 = Menge von Testmengen aus $\Omega$ \\
 = Z\textsuperscript{$\Omega$} \\
 3) WSK-Ma$\ss$ P: \\
 P: $\Sigma$ $\to$ [0,1]\\
 \\
 \underline{Def.:} WSK-Ma$\ss$ P: $\Sigma$ $\to$ [0,1]\\
 P[$\Omega$] = 1 \\
 P[$\emptyset$] = 0 \\
 F\"ur alle paarweise disjunkten Ereignisse E\textsubscript{1}, E		\textsubscript{2}, \ldots \\ 	
 P[$\bigcup$ E\textsubscript{i}] = $\Sigma$(P[E\textsubscript{i}]) \\ 	\\
 Ereignisse sind Mengen \\
 E\textsubscript{1}  $\cap$ E\textsubscript{2}: E\textsubscript{1} und E\textsubscript{2} treten gleichzeitig auf \\
 E\textsubscript{1}  $\cup$ E\textsubscript{2}: mindestens eines von E	\textsubscript{1} und E\textsubscript{2} tritt auf \\
 $\overline{E}$: Komplement von E \\
 \\
 \underline{Lem.:} P[E\textsubscript{1} $\cup$ E\textsubscript{2}] = P[E\textsubscript{1}] + P[E\textsubscript{2}] - P[E\textsubscript{1} $\cap$ E\textsubscript{2}] \\
 \underline{Bew.:} \\
 P[E\textsubscript{1}] = P[E\textsubscript{1}$\setminus$(E\textsubscript{1} $\cap$ E\textsubscript{2})] + P[E\textsubscript{1} $\cap$ E\textsubscript{2}] \\ 
 P[E\textsubscript{2}] = P[E\textsubscript{2}$\setminus$(E\textsubscript{1} $\cap$ E\textsubscript{2})] + P[E\textsubscript{1} $\cap$ E\textsubscript{2}] \\
 P[E\textsubscript{1} $\cup$ E\textsubscript{2}] = P[E\textsubscript{1}$\setminus$(E\textsubscript{1} $\cap$ E\textsubscript{2})] + P[E\textsubscript{2}$\setminus$(E\textsubscript{1} $\cap$ E\textsubscript{2})] + P[E\textsubscript{1} $\cap$ E\textsubscript{2}]
 \\
 \underline{Lem.:} Union Bound \\
 F\"ur alle Ereignisse E\textsubscript{1}, E\textsubscript{2}, \ldots \\
 P[$\bigcup$ E\textsubscript{i}] $\le$ $\Sigma$(P[E					\textsubscript{i}]) \\
 \\
 \underline{Lem.:} Inklusion und Exklusion \\
 F\"ur Ereignisse E\textsubscript{1}, E\textsubscript{2}, ...\\
 P[$\bigcup$ E\textsubscript{i}] = $\Sigma$(P[E\textsubscript{i}]) - $\Sigma$(P[E\textsubscript{i} $\cap$ E\textsubscript{j}]) + $\Sigma$		(P[E\textsubscript{i} $\cap$ E\textsubscript{j} $\cap$ E\textsubscript{k}]) + ... + (-1)\textsuperscript{l+1} $\Sigma$(P[E\textsubscript{i} $\cap$ E\textsubscript{j} $\cap$ \ldots $\cap$ E		\textsubscript{l}]) \\
 \\
 \underline{Algorithmus: \"Aquivalenz von Polynomen} \\
 (x-1)(x+2)(x-3)(x+4)(x-5)(x+6) \\
 $\stackrel{\mathrm{?}}=$ \\
 x\textsuperscript{6} + 3x\textsuperscript{5} - 14x\textsuperscript{4} - 78x\textsuperscript{3} + 400x\textsuperscript{2} + 444x - 720 \\
 F(x) als Produkt: Ausmultiplizieren $\to$ $\theta$(d\textsuperscript{2}) 	Operationen \\
 G(x) als Normalform \\
 \\
 Allg. F(x) $\stackrel{\mathrm{?}}\equiv$ G(x) \\
 \\
 PolyEq(F(x), G(x))) \\
 r $\stackrel{\mathrm{R}}\leftarrow$ \{1, \ldots, 1000d\} \\
 \underline{if} F(x) $\neq$ G(x) \underline{then} \\
 return ''different'' \\
 \underline{else} \\
 return ''maybe equal'' \\
 \\
 falls: \\
 F(x) = G(x) $\wedge$ ''maybe equal'': die Antwort ist korrekt \\
 F(x) $\neq$ G(x) $\wedge$ ''different'': die Antwort ist korrekt \\
 F(x) $\neq$ G(x) $\wedge$ ''maybe equal'': die Antwort ist falsch!\\
 \\
 Wahrscheinlichkeit, dass Antwort falsch ist: \\
 D(x) = F(x) - G(x) $\Leftrightarrow$ falls r Nullstelle von D(x) \\
 Grad d $\Leftrightarrow$ h\"ochstens d Nullstellen\\
 WSK = $\frac{d}{1000 \cdot d}$ = $\frac{1}{1000}$ \\
 \\
 \underline{Randomisierte Algorithmen} \\
 \\
 Las Vegas Algorithmus: \\
 terminiert eventuell nicht \\
 Ergebnis immer korrekt \\
 \\
 Monte Carlo Algorithmus: \\
 terminiert immer \\
 Ergebnis eventuell falsch \\
 $\rightarrow$ F\"ur Entscheidungsprobleme (YES/NO) \\
 Einseitige Fehler \\
 Eine Antwort (YES/NO) ist immer korrekt \\
 Zweiseitige Fehler \\
 \ldots \\
 \\
 \underline{Unabh\"angigkeit} \\
 \underline{Def.:} Ereignisse E und F sind unabh\"angig $\leftrightarrow$ P[E $\cap$ F] = P(E) $\cdot$ P(F) \\
 \\
 \underline{bedingte Wahrscheinlichkeit} \\
 P[E$\mid$F] = $\frac{P[E \cap F]}{P(F)}$ \\
 \\
 Algorithmus MultiPolyEq(\ldots) \\
 \underline{for} j = 1, \ldots, k \underline{do} \\
 \underline{if} PolyEq(\ldots) = ''different'' \underline{then} \\
 return ''different'' \\
 \underline{end} \\
 return ''maybe equal'' \\
 \\
 Wahrscheinlichkeiten sind unabh\"angig \\
 P[eine Runde falsch] = $\frac{1}{1000}$ \\
 P[alle k Runden falsch] = ($\frac{1}{1000}$)\textsuperscript{k} \\
 \\
 \underline{Theorem} \\
 E\textsubscript{1}, E\textsubscript{2}, ..., E\textsubscript{k} paarweise disjunkte Ereignisse \\
 $\Omega$ = $\bigcup$ E\textsubscript{i} \\
 P[A] = $\Sigma$ P[A$\mid$E\textsubscript{j}] $\cdot$ P[E			\textsubscript{j}] \\
 \\
 \textbf{\underline{Theorem von Bayes}} \\
 E\textsubscript{1}, E\textsubscript{2}, ..., E\textsubscript{k} paarweise disjunkte Ereignisse \\
 $\Omega$ = $\bigcup$ E\textsubscript{i} \\
 F\"ur alle A, E\textsubscript{j}: P[E\textsubscript{j}$\mid$A] = $		\frac{P[E\textsubscript{j} \cap A]}{P(A)}$ = $\frac{P[A \mid E			\textsubscript{j}] \cdot P[E\textsubscript{j}]}{\Sigma P[A \mid E		\textsubscript{j}] \cdot P[E\textsubscript{j}]}$ \\
 \\
 \underline{Intrusive-Prevention-System} \\
 Soll Alarm (A) geben, falls eine Intrusion (I) vorliegt \\
 P[A$\mid$I] = 0.95 \\
 P[A$\mid$T] = 0.01 \\
 P[I] = 0.02 \\
 \\
 WSK daf\"ur, dass bei Alarm (A) tats\"achlich eine Inklusion passiert: \\
 P[I$\mid$A] = 0.66 \\
 
 	
 \end{document}