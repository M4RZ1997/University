\documentclass{article}
\usepackage{geometry}
\usepackage{paralist}
\usepackage[T1]{fontenc}
\usepackage{reledmac}
\usepackage{changepage}
\usepackage{amsmath}
\usepackage{scalerel,amssymb}

\usepackage{pgfplots}
\usepackage{tikz}
\usetikzlibrary{positioning}
\usetikzlibrary{shapes.geometric, arrows}
\tikzstyle{arrow} = [thick,->,>=stealth]

\usepackage{fancyhdr}
\fancyhead[L]{
	\begin{tabular}{l}
		\LARGE \textbf{\textsc{Distributed Algorithms}} \\
		\Large Exercise 06
	\end{tabular}
}
\fancyhead[R]{
	\begin{tabular}{r}
		16-124-836 \\
		Marcel \textsc{Zauder}
	\end{tabular}
}
\renewcommand{\headrulewidth}{0.4pt}
\fancyfoot[C]{\thepage}
\renewcommand{\footrulewidth}{0.4pt}

\usepackage{hyperref}

\begin{document}
	\pagestyle{fancy}
	\hfill
	\section*{6.1 \textsc{Atomic} Register Execution}
	\begin{adjustwidth}{2em}{2em}
		Let processes $\{p, q, r, s, t\}$ have \textbf{ranks} $\{1, 2, 3, 4, 5\}$ respectively.
		\subsection*{(a) $read_r() \rightarrow x$ and $read_s() \rightarrow y$}
		\begin{adjustwidth}{2em}{2em}
			First we can say that $x$ has been written immediately after \textit{write}$_p(x)$ has started. Therefore $r$, $s$ and $t$ have stored $x$ when the $\textit{read}_r()$ operation terminates hence it returns $x$. Before $\textit{read}_s()$ operation is executed the process $s$ will store the value of $y$ because although the timestamps of both \textit{write} operation is equal to the \textbf{rank} of $q$ is higher. Therefore process $s$ can return $y$. Because the $\textit{read}_t()$ operation is called after the \textit{write} operations have finished and process $q$ has a higher rank than process $p$ it will also return $y$.
		\end{adjustwidth}
		\subsection*{(b) $read_r() \rightarrow y$ and $read_s() \rightarrow x$}
		\begin{adjustwidth}{2em}{2em}
			First we can say that $x$ has been written immediately after \textit{write}$_p(x)$ has started. Therefore $r$, $s$ and $t$ have stored $x$ when the $\textit{read}_s()$ operation terminates hence it returns $x$. Before $\textit{read}_r()$ operation is executed the process $r$ will store the value of $y$ because although the timestamps of both \textit{write} operation is equal to the \textbf{rank} of $q$ is higher. Therefore process $r$ can return $y$. Because the $\textit{read}_t()$ operation is called after the \textit{write} operations have finished and process $q$ has a higher rank than process $p$ it will also return $y$ (same argumentation as above, just switched $r$ and $s$).
		\end{adjustwidth}
	\end{adjustwidth}
	
	\section*{6.2 Erasure-Coded Storage}
	\begin{adjustwidth}{2em}{2em}
		\subsection*{(a) Erasure Code Picking}
		\begin{adjustwidth}{2em}{2em}
			A suitable erasure code for a distributed storage system with at most $\frac{n}{2}$ crashed nodes could be any $(\lceil \frac{n}{2} \rceil , n)$ erasure code. A good abstraction of such an erasure code could be the following \textit{Reed-Solomon} code: \\
			Let $k := \lceil \frac{n}{2} \rceil$. The message $m$ which should be stored is split up into bit-strings $m_i$ of length $k$. For each of these bit-strings $m_i := (x_1, x_2, \ldots x_k)$ we compute the Lagrange polynomial $p$ such that $p(i - 1) = x_i$. These values $p(i)$ are then stored at node number $i \mid 0 \leq i \leq n-1$. \\
			An erasure code implemented like this with $(int) f < \frac{n}{2}$ failed nodes, it holds that $n-f \geq \lceil \frac{n}{2} \rceil$, which is a sufficient number of fragments to reconstruct $m_i$. \\
			The storage efficiency of any $(\lceil \frac{n}{2} \rceil , n)$ erasure code is $\approx 50 \%$.
		\end{adjustwidth}
		\newpage
		\subsection*{(b) Modify majority voting \textsc{regular} register to an erasure-coded \textsc{safe} register}
		\begin{adjustwidth}{2em}{2em}
			The Algorithm 4.2 can be adjusted as the following. Not changed event handlings are left out and are similar as in the algorithm. Furthermore we assume that a suitable lagrange interpolation is available from an external source. \\
			\begin{center}
				\begin{tabular}{l}
					\underline{Implements:} \\
					\ \ $(1,N)$ \textsc{safe} register, \textit{\textbf{instance}} \textit{onsr} \\
					\\
					\underline{upon} $\langle \textit{onsr, \textsc{Write}} \mid v \rangle$ \underline{do} \\
					\ \ $v = (v_1, v_2, \ldots , v_k)$ , \textbf{whereas} $k = \frac{N}{2}$ \\
					\ \ $wts := wts + 1$ \\
					\ \ $acks := 0$ \\
					\ \ $p := \textit{lagrange\_interpolation} ((0, v_1), \ldots , (k-1, v_k)) $ \\
					\ \ \underline{for} $i = 1$ to $N$: \\
					\ \ \ \ trigger $\langle \textit{pl}, \textsc{send} \mid p_i, [\textsc{Write}, wts, ((i-1), p(i-1))] \rangle$ \\
					\\
					\underline{upon} $\langle \textit{pl}, \textsc{deliver} \mid q, [\textsc{Value}, r, ts', v'] \rangle$ s.t. $r =rid$ \underline{do} \\
					\ \ $readlist[q] := (ts', v')$ \\
					\ \ \underline{if} $\mid readlist \mid > N/2$ \underline{then} \\
					\ \ \ \ $p := \textit{reconstruct\_polynom}(readlist)$ \\
					\ \ \ \ \underline{for}  $i := 1$ to $N/2$: \\
					\ \ \ \ \ \ $v_i := p(i - 1)$ \\
					\ \ trigger $\langle \textit{onsr}, \textsc{ReadReturn} \mid (v_1 \| \cdots \| v_{k}) \rangle$
				\end{tabular}
			\end{center}
		\end{adjustwidth}
		\subsection*{(c) Why is it difficult to extend this protocol to regular semantics?}
		\begin{adjustwidth}{2em}{2em}
			\begin{quote}
				\textbf{\textsc{regular}:} A read() not concurrent with a write returns the most recently written value. Otherwise read() returns the most recently written value or the concurrently written value.
			\end{quote}
			\hfill \\
			For the erasure code register the information of any value $v$ is spread across $n$ different registers, whereas none of these alone has sufficient information to reconstruct that value $v$. Becuase of the need of information from multiple different registers to reconstruct $v$, it is not trivial to ensure that the read is consistent, which would lead to a \textsc{regular} behaviour.
		\end{adjustwidth}
	\end{adjustwidth}
\end{document}